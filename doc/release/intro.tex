\chapter{Introduction}

This manual is intended for driving the release process of \FramaC
distribution. It covers several topics: people involved in the release effort,
repository resources and web site maintenance. The last chapter provides the
ultimate procedure for releasing a new distribution of \FramaC.

\section{Roles}

Some individuals have particular expertises in some fields: they are indicated
like \expertise{this} whenever appropriated.

The members of \FramaC team involved in the release effort play different
roles, and we must distinguish between:

\begin{description}
\item[Developers:] including the kernel and plug-ins developers. They
  are responsible for all the source files in the repository, and the
  associated documentation.
%% \item[Binary Builders:] they are responsible for creating a binary distribution
%%   of the release for a specific architecture \expertise{none currently}.
\item[Web Site Maintainers:] they are responsible for updating the web site,
  during the release and for possible later updates \expertise{Allan, Augustin}.
\item[Release Manager:] they are responsible for the organisation of the
  release process
\end{description}

\section{Release Overview}

A \FramaC release consists of the following main steps:
\begin{enumerate}

\item \textbf{The branch stage (Chapter~\ref{chap:branch}).} A branch is
  created, in which the release  will be prepared. It will ultimately become the
  released version. The development of unstable features may continue in master,
  while the branch is dedicated to the ongoing release.

\item \textbf{The validation stage (Chapter~\ref{chap:validation}).} When we
  estimate that \FramaC is ready to be released, the bug tracking system is
  updated to reflect the distribution state. All artifacts that must be
  distributed are generated in the continuous integration and validated by the
  Release Manager. The version commit is created and tagged.

\item \textbf{The deployment stage (Chapter~\ref{chap:deploy}).} The Release
  Manager starts the release pipeline in the continuous integration. The last
  stage releases the branch, the GitLab release, the website, the wiki and
  creates an opam repository branch on the Frama-C organisation on GitHub. Time
  to send an email on the Frama-C discuss mailing list, toot and spread the
  world!

\end{enumerate}



%%%Local Variables:
%%%TeX-master: "release"
%%%End:
