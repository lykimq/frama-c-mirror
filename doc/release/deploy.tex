\chapter{Deployment stage}
\label{chap:deploy}

The validation stage (Chapter~\ref{chap:validation}) must be complete before
starting this stage. The tasks listed in this section are mostly performed by
the Release Manager.

\section{Release pipeline}
\label{sec:release-pipeline}

Go to the Build $\rightarrow$ Pipelines section of GitLab. Start the release pipeline:
\begin{itemize}
  \item click ``Run Pipeline''
  \item select the stable branch
  \item add the variable \texttt{RELEASE} with value \texttt{yes}
  \item run the pipeline.
\end{itemize}

All tests will be run, as well as the \texttt{opam-pin} target and the job that
create the release artifacts. The most important for the remaining tasks are the
following:
\begin{description}
\item[\texttt{release-content}] has all generated artifacts
\item[\texttt{release-website}] creates a merge request assigned to
  you on the website repository,
\item[\texttt{release-wiki}] creates a wiki page on the public
  Frama-C repository (you can see a link in the Activity section),
\item[\texttt{release-opam}] creates the branch on the opam-repository
  of the Frama-C organization on GitHub for Opam release,
\item[\texttt{release-branch}] is ready to push the stable branch on
  the public repo (it must be started manually)
\item[\texttt{release-create}] is ready to create the GitLab release
  on \texttt{pub} (it must be started manually and requires the branch).
\end{description}

\textbf{Important note:}~If the version is a whole number (\texttt{X0.0}),
a special step must be taken before starting \texttt{release-create}: add
\texttt{X*} as a tag that maintainers can create on \texttt{pub/frama-c}.

After running \texttt{release-branch} and \texttt{release-create}, either the
tag indicates a beta release and then this tag is pushed on the public GitLab
repository (\url{https://git.frama-c.com/pub/frama-c/-/tags}), or it is a final
release and the release should be available in
\url{https://git.frama-c.com/pub/frama-c/-/releases}, as well as the tag of the
version in \url{https://git.frama-c.com/pub/frama-c/-/tags}.

\textbf{Important note:}~If for some reason the pipeline fails and you push new
commits to the branch to fix the issue, do not forget to update the tag created
at the end of Chapter.~\ref{chap:validation}, and to force-push this new reference.

\section{Check the website}
\label{sec:check-website}

Once the pipeline for the website has run, open \url{https://pub.frama-c.com}.

\textbf{Note:} \texttt{https://pub.frama-c.com/} serves the content of the latest
commit successfully compiled by the CI. If there is a delay between the moment where
you launch the pipeline and the moment you inspect the website, it may not be up-to-date.
You can either serve the content locally on your machine (instructions are provided
on the README of the website's repo), or by re-running the pages jobs on the relevant pipeline

\begin{itemize}
  \item \texttt{index.html} must display:
  \begin{itemize}
    \item the new event,
    \item if it is beta a link to the beta next to "All versions".
  \end{itemize}
  \item \texttt{/fc-versions/<codename>.html}:
  \begin{itemize}
    \item check Changelog link,
    \item check manual links (reminder: the links are dead at this moment), it must contain \texttt{NN.N-Version}
    \item check ACSL version.
  \end{itemize}
  \item \texttt{/html/changelog.html\#<codename>-<NN.M>}
  \item \texttt{/html/acsl.html}: check ACSL versions list
  \item \texttt{rss.xml}: check last event
\end{itemize}

On GitLab, in the website repository, in the merge request assigned to you, the
following files must appear as \textbf{new} in \texttt{download}:
\begin{itemize}
  \item \texttt{acsl-<X.YY>.pdf}
  \item \texttt{acsl-implementation-<NN.M>-<Codename>.pdf}
  \item \texttt{plugin-development-guide-<NN.M>-<Codename>.pdf}
  \item \texttt{user-manual-<NN.M>-<Codename>.pdf}
  \item \texttt{<plugin>-manual-<NN.M>-<Codename>.pdf}\\
        for Aorai, EVA, Metrics, RTE and WP
  \item \texttt{e-acsl/e-acsl-<X.YY>.pdf}
  \item \texttt{e-acsl/e-acsl-implementation-<NN.M>-<Codename>.pdf}
  \item \texttt{e-acsl/e-acsl-manual-<NN.M>-<Codename>.pdf}
  \item \texttt{aorai-example-<NN.M>-<Codename>.tar.gz}
  \item \texttt{frama-c-<NN.M>-<Codename>.tar.gz}
  \item \texttt{frama-c-<NN.M>-<Codename>-api.tar.gz}
  \item \texttt{frama-c-server-<NN.M>-<Codename>-api.tar.gz}
  \item \texttt{hello-<NN.M>-<Codename>.tar.gz}
\end{itemize}

For a final release \textbf{ONLY}, the following files must appear as \textbf{modified} in \texttt{download}:

\begin{itemize}
  \item \texttt{acsl.pdf}
  \item \texttt{frama-c-acsl-implementation.pdf}
  \item \texttt{frama-c-plugin-development-guide.pdf}
  \item \texttt{frama-c-api.tar.gz}
  \item \texttt{frama-c-server-api.tar.gz}
  \item \texttt{frama-c-user-manual.pdf}
  \item \texttt{frama-c-<plugin>-manual.pdf}\\
        for Aorai, Eva, Metrics, RTE, and WP
  \item \texttt{frama-c-value-analysis.pdf}
  \item \texttt{e-acsl/e-acsl.pdf}
  \item \texttt{e-acsl/e-acsl-implementation.pdf}
  \item \texttt{e-acsl/e-acsl-manual.pdf}
  \item \texttt{frama-c-aorai-example.tar.gz}
  \item \texttt{frama-c-hello.tar.gz}
\end{itemize}

If everything is fine, merge the website and ask a website maintainer to put it
online (\expertise{Allan, Augustin}).

\section{Public GitLab}
\label{sec:public-gitlab}

Open the generated wiki page (visible in the public website activity). Check the
links (files are available only once the website has been put online). If
everything is fine, edit the main wiki page and the sidebar with a link to the
page (\url{https://git.frama-c.com/pub/frama-c/-/wikis/home}).

\section{External Plug-in Release}
\label{sec:ext-plug-ins-release}

If a release of Frama-Clang and MetAcsl is scheduled together
with the main Frama-C release, you can now launch the release pipelines
on their respective repositories (\texttt{frama-c/frama-clang} and
\texttt{frama-c/meta}), with the same protocol as for the main
Frama-C repository described in Sect.~\ref{sec:release-pipeline}.
The only difference is that there is only a single \texttt{release}
\todo{it would be good to split the script in several sub-jobs} task.
Then, check the website MR that have been created (one for each
plug-in), and merge them if everything is alright.

In particular, the \texttt{\_fc-plugins/frama-clang.md} needs manual
update\todo{reword this page to avoid this manual step}.

Note that these releases can be done at a later point and independently from
Frama-C. If you happen to release a plug-ins after the PR for Frama-C has been
merged on opam repository (see Sect.~\ref{sec:opam-package}), a new PR will be
opened for the plug-in. Otherwise, the same branch will be used to create all
the packages.

\section{Announcements}
\label{sec:announcements}

\begin{itemize}
\item Send an e-mail to \texttt{frama-c-discuss} announcing the release.
\item Toot the release (\url{https://fosstodon.org/@frama_c}),
  pointing to the Downloads page.
\item Ideally, a blog post should arrive in a few days, with some interesting
  new features.
\end{itemize}

\section{Opam package}
\label{sec:opam-package}

You'll need a GitHub account to create a pull request on the official opam
repository\footnote{\texttt{ocaml/opam-repository.git}}. Go to the \FramaC
GitHub organization opam repository (\url{https://github.com/Frama-C/opam-repository}).
Find the branch corresponding to the release and create the pull-request on the
official opam repository.

As mentioned in Sect.~\ref{sec:ext-plug-ins-release}, the PR can contain
plug-in(s) package(s) in addition to the Frama-C package itself if plug-ins
have also been released.

Once the PR has been merged, make sure that the branch is removed from
the fork in
\texttt{Frama-C/opam-repository}\todo{If possible in github's REST API,
make sure that the merge will erase the branch automatically}.

\section{Other repositories to update}
\label{sec:other-repos-update}

Check if other \FramaC (and related) repositories need to be updated:

\begin{itemize}
\item \texttt{acsl-language/acsl} (if last minute patches were applied)
\item \texttt{pub/open-source-case-studies} (\expertise{André})
\begin{itemize}
\item update the reference commit of the frama-c submodule to the tag of the release
\item \texttt{make framac}
\item \texttt{make clean}
\item \texttt{make -j $(nproc) all}
\end{itemize}
\item \texttt{pub/sate-6} (\expertise{André})
\item other \texttt{pub} repositories related to Frama-C...\todo{which ones?}
\end{itemize}

\section{Docker image preparation}
\label{sec:dock-image-prep}

This section only applies to non-beta releases.

\textbf{Note:} you need access to the \texttt{framac} Docker Hub account to be
able to upload the image.

Copy the \texttt{.tar.gz} archive to the \texttt{dev/docker} directory.

Run:
\begin{lstlisting}
make FRAMAC_ARCHIVE=<the_archive> custom.debian
\end{lstlisting}
It should decompress the archive, build and test the Frama-C Docker image.

If the local tests do not work, check that the OCaml version and package
dependencies are all up-to-date.

If the image is built succesfully, you can also try building the GUI image and
the stripped image:
\begin{lstlisting}
make FRAMAC_ARCHIVE=<the_archive> custom-gui.debian
make FRAMAC_ARCHIVE=<the_archive> custom-stripped.debian
\end{lstlisting}

If you want to upload these images to the Docker Hub, you can re-tag them and
upload them, e.g.

\begin{lstlisting}
make -C \
  TAG="framac/frama-c:custom.debian" \
  AS="framac/frama-c:<VERSION>" \
  push

make -C \
  TAG="framac/frama-c:custom-stripped.debian" \
  AS="framac/frama-c:<VERSION>-stripped" \
  push

make -C \
  TAG="framac/frama-c:custom-gui.debian" \
  AS="framac/frama-c-gui:<VERSION>" \
  push
\end{lstlisting}

Where \texttt{<VERSION>} is the release number, possibly with a suffix, but
{\em without} characters such as \texttt{+}. For instance, you can use
\texttt{23.1-beta} for a beta release.

You will need to have setup your \texttt{framac} Docker Hub account for this to work.

%%% Local Variables:
%%% mode: latex
%%% TeX-master: "release"
%%% End:
