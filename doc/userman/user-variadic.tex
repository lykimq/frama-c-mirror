%% --------------------------------------------------------------------------
%% --- Variadic Plugin
%% --------------------------------------------------------------------------

\chapter{Variadic Plug-in}
\label{user-variadic}

This chapter briefly presents the \textttdef{Variadic} plug-in, which
performs the translation of calls to variadic functions into calls to
semantically equivalent, but non-variadic functions.

\section*{Variadic functions}

Variadic functions accept a variable number of arguments, indicated in
their prototype by an ellipsis (\ldots) after a set of fixed arguments.

Some functions in the C standard library are variadic, in particular
formatted input/output functions, such as \texttt{printf}/\texttt{scanf}.
Due to the dynamic nature of their arguments, such functions present additional
challenges to code analysis. The \textttuse{Variadic} helps dealing with some
of these challenges, reducing or eliminating the need for plug-ins to have
to deal with these special cases.

\section{Translating variadic function calls}

ACSL does not allow the direct specification of variadic functions: variadic
arguments have no name and no statically known type. The \textttuse{Variadic}
plug-in performs a semantically-preserving translation of calls to such
functions, replacing them with non-variadic calls.

For instance, consider the following user-defined variadic function \verb+sum+,
whose first argument \verb+n+ is the number of elements to be added, and the
remaining \verb+n+ arguments are the values themselves:

\begin{ccode}
#include <stdarg.h> // for va_* macros
int sum(unsigned n, ...) {
  int ret = 0;
  va_list list;
  va_start(list, n);
  for(int i = 0; i < n; i++){
    ret += va_arg(list, int);
  }
  va_end(list);
  return ret;
}
\end{ccode}

\begin{ccode}
int main(){
  return sum(5, 6, 9, 14, 12, 1);
}
\end{ccode}

Since \textttuse{Variadic} is enabled by default, running \FramaC on this code
will activate the variadic translation. The main differences in the translated
code are:

\begin{itemize}
\item the prototype of \verb+sum+ becomes
  \verb+int sum(unsigned n, void * const *__va_params)+;
\item the call to \verb+sum+ is converted into:
  \begin{ccode}
  int sum_ret; // temporary storing the return value
  {
    int __va_arg0 = 6;
    int __va_arg1 = 9;
    int __va_arg2 = 14;
    int __va_arg3 = 12;
    int __va_arg4 = 1;
    void *__va_args[5] = {& __va_arg0, & __va_arg1,
                            & __va_arg2, & __va_arg3, & __va_arg4};
    sum_ret = sum(5,(void * const *)(__va_args));
  }
  \end{ccode}
\end{itemize}

This translation is similar to the relation between functions such as
\verb+printf+ and \verb+vprintf+, where the former accepts a variable number of
arguments, while the latter accepts a single \verb+va_list+ argument.

\section{Automatic generation of specifications for libc functions}

The most common use case of variadic functions are the ubiquitous
\verb+printf+/\verb+scanf+, but a few other functions in the standard
C library are variadic, such as \verb+open+ and \verb+fcntl+.
The former are entirely dependent on the {\em format} string, which can have
any shape, while the latter are limited to a fixed set of possible argument
numbers and types. In both cases, it is possible to specialize the function
call and generate an ACSL specification that (1) performs some checks for
undefined behaviors (\eg that the argument given to a \verb+%d+ format is a
\verb+signed int+), and (2) ensures postconditions about the return value
and modified arguments (namely for \verb+scanf+). The \textttuse{Variadic}
plug-in generates such specifications whenever possible.

Note that not all calls have their specification automatically generated;
in particular, calls to formatted input/output functions using non-static
format strings are not handled, such as the following one:

\begin{ccode}
  printf(n != 1 ? "%d errors" : "%d error", n_errors);
\end{ccode}

In this case, the variadic translation is performed, but the function call
is not specialized, and no specification is automatically generated.

\section{Usage}

\subsection{Main options}

By default, \textttuse{Variadic} is enabled and runs after parsing.
A few options are available to modify this behavior:

\begin{description}

\item \texttt{\optiondef{-}{variadic-no-translation}} : disables the translation
  performed by the plug-in; to be used in case it interferes with some other
  plug-in or analysis.
\item \texttt{\optiondef{-}{variadic-no-strict}} : disables warnings about
  non-portable implicit casts in the calls of standard variadic functions,
  \ie casts between distinct integral types which have the same size and
  signedness.

\end{description}

\subsection{Similar diagnostics by other tools}

Some of the issues detected by \textttuse{Variadic}, namely some kinds of
incompatible arguments in formatted input/output functions, are also detected
by compilers such as GCC and Clang, albeit such diagnostics are rarely enabled.

In particular, GCC's option \verb+-Wformat-signedness+ (available from GCC 5)
reports some issues with signed format specifiers and unsigned arguments, and
vice-versa, in cases such as the following:

\begin{ccode}
  printf("%u", -1);
\end{ccode}

Clang's option \verb+-Wformat-pedantic+ (available at least since Clang 4.0,
possibly earlier) also enables some extra diagnostics:

\begin{ccode}
  printf("%p", "string");
\end{ccode}

\begin{logs}
  warning: format specifies type 'void *' but the argument has type 'char *'.
\end{logs}

Note that no single tool is currently able to emit all diagnostics emitted
by the other two.

\subsection{Common causes of warnings in formatted input/output functions}

Many C code bases which make use of formatted input/output functions do not
specify all of them in a way that is strictly conformant to the C standard.
This may result in a rather large number of warnings emitted by
\textttuse{Variadic}, not all of them immediately obvious.

It is however important to remember that C99, §7.19.6.1 states that:

\begin{quote}
  If a conversion specification is invalid, the behavior is undefined.
  If any argument is not the correct type for the corresponding conversion
  specification, the behavior is undefined.
\end{quote}

CERT C lists this as Rule FIO47-C.

Some common types of discrepancies are listed below, with an explanation
of their causes.

\paragraph{Usage of \texttt{\%u} or \texttt{\%x} for values of type
  \texttt{unsigned char} and \texttt{unsigned short}.}

The warning emitted by \textttuse{Variadic} in this case will mention a
{\em signed} type being cast to {\em unsigned}.
Albeit counterintuitive, this is a consequence of the fact that default
argument promotions take place for variadic arguments, and thus
\verb+unsigned char+ and \verb+unsigned short+ arguments are promoted to
(signed) \verb+int+, which is incompatible with \verb+%u+ and \verb+%x+.

To avoid the warning, use the appropriate length modifiers:
\verb+hh+ for \verb+char+ and \verb+h+ for \verb+short+.

\paragraph{Usage of \texttt{\%o}, \texttt{\%x} and \texttt{\%X} to print
  signed values.}

The standard specifies that all of these modifiers expect unsigned arguments.
A cast to the corresponding unsigned type must therefore be present.

\subsection{Pretty-printing translated code}

The output produced by \textttuse{Variadic}, in particular when using
\verb+va_*+ macros (such as \verb+va_list+), is not guaranteed to be
parsable, unless option \verb+-print-libc+ is enabled.
