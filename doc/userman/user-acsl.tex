\chapter{ACSL Extensions} % here we do not use the macro (avoids a warning)
\label{cha:acsl-extensions}
\section{Extension Syntaxes}
\label{acsl:syntax}

When a plug-in registers an extension, it can be used in \acsl annotations
with 2 different syntaxes. As an example, with an extension \lstinline|bar|
registered by the plug-in \lstinline|foo|, we can use the short syntax
\lstinline|bar _| or the complete syntax \lstinline|\foo::bar _|.

The complete syntax is useful to print better warning/error messages, and to
better understand which plug-in introduced the extension. Additionnaly, all
extensions coming from an unloaded plug-in can be ignored this way. For
example, if \Eva is not loaded, \lstinline|\eva::unroll _| annotations will be
ignored with a warning, whereas \lstinline|unroll _| cannot be identified as
being supported by Eva, which means that it can only be treated as a user
error.

\section{Handling Indirect Calls with \texttt{calls}}
\label{acsl:calls}

In order to help plug-ins support indirect calls (i.e. calls through
a function pointer), an \acsl extension is provided. It is introduced
by keyword \lstinline|calls| and can be placed before
a statement with an indirect call to give the list of
functions that may be the target of the call. As an example,
\begin{ccode}
/*@ calls f1, f2, ... , fn */
*f(args);
\end{ccode}
indicates that the pointer \lstinline|f| can point to any one of
\lstinline|f1|, \lstinline|f2|, ..., \lstinline|fn|.

It is in particular used by the WP plug-in (see \cite{wp} for more information).

\section{Importing External Module Definitions}
\label{acsl:modules}

Support for \acsl modules has been introduced in \nextframacversion.
Module definitions can be nested. Inside a module \verb+A+,
a sub-module \verb+B+ will actually defines the module \verb+A::B+, and so on.

Notice than for long identifiers like \verb+A::B::C+ to be valid, no space is
allowed around the \verb+:+ characters, and \verb+A+, \verb+B+, \verb+C+ must be
regular \acsl identifiers, i.e.~they shall only consist of upper case or lower
case letters, digits, underscores, and must start with a letter.

Inside module \verb+M+ declaration, where \verb+M+ it the long identifier of the
module being declared, a logic declaration \verb+a+ will actually define the
symbol \verb+M::a+. You shall always use the complete name of an identifier to
avoid ambiguities in your specifications. However, in order to ease reading, it
is also possible to use shortened names instead.

The rules for shortening long identifiers generalize to any depth of nested
modules. We only illustrate them in a simple case. Consider for instance a logic
declaration \verb+a+ in module \verb+A::B+, depending on the context, it is
possible to shorten its name as follows:
\begin{itemize}
\item Everywhere, you can use \verb+A::B::a+;
\item Inside module \verb+A+, you can use \verb+B::a+;
\item Inside module \verb+A::B+, you can use \verb+a+;
\item After annotation \lstinline[language=ACSL]|import A::B|, you can use
  \verb+B::a+;
\item After annotation \lstinline[language=ACSL]|import A::B as C|, you can use
  \verb+C::a+;
\end{itemize}

You may also use local \lstinline[language=ACSL]+import+ annotations inside
module definitions, in which case the introduced aliases will be only valid
until the end of the module scope.

Depending on dedicated plug-in support, you may also import modules definitions
from external specifications, generally from an external proof assistant like
\textsf{Coq} or \textsf{Why3}. The \acsl extended syntax for importing external
specifications is as follows:

\begin{lstlisting}[language=ACSL]
import <Loader>: <ModuleName> [ \as <Name> ];
\end{lstlisting}

This is a generalization of the regular \acsl \lstinline[language=ACSL]|import|
clause just mentioned above. The \verb+<Loader>+ name identifies the kind of
external specifications to be loaded. Loaders are defined by dedicated plug-in
support only, and you shall consult the documentation of each plug-in to known
which loaders are available. A loader syntax can be either just a name, used
when the extension was registered, or \verb+<\plugin::name>+. The second syntax
is useful to avoid ambiguities if several plug-ins register a module importer
extension with the same name.

The \verb+<ModuleName>+ identifies both the name of the imported module and the
external specification to be imported, with a \verb+<Loader>+ dependent meaning.

The alias name \verb+<Name>+, if provided, has the same meaning than when
importing regular module names (just described above) in the current scope.

When importing \emph{external} specifications, depending on the \verb+<Loader>+
used, it is possible to have logic identifiers with an extended lexical format:
\begin{itemize}
\item \verb+M::(op)+ where \verb+M+ is a regular module identifier, and
  \verb+op+ any combination of letters, digits, operators, brackets, braces,
  underscores and quotes. For instance, \verb+map::Map::([<-])+ is a syntactically
  valid identifier, and \verb+number::Complex::(<=)(a,b)+ is a syntactically valid
  expression.
\item \verb+M::X+ where \verb+M+ is a regular module identifier and
  \verb+X+ any combination of letters, digits, underscores and quotes.
  For instance, \verb+Foo::bar'jazz+ is a syntactically valid identifier.
\end{itemize}

External module importers are defined by plug-ins \textit{via} the extension API,
consult the plug-in manuals and the plug-in developer manual for more details.

%%% Local Variables:
%%% mode: latex
%%% TeX-master: "userman"
%%% End:
