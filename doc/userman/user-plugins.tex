\chapter{Setting Up Plug-ins}
\label{user-plugins}

The \FramaC platform has been designed to support third-party plug-ins. In the
present chapter, we present how to configure, compile, install, run and update
such extensions. This chapter does not deal with the development of new plug-ins (see the \textsf{Plug-in Development
  Guide}~\cite{plugin-dev-guide}). Nor does it deal with usage
of plug-ins, which is the purpose of individual plug-in documentation (see e.g.~\cite{value,wp,slicing}).

\section{The Plug-in Taxonomy}\label{sec:plugin-taxonomy}

There are two kinds of plug-ins: \emph{internal} and \emph{external} plug-ins.
\index{Plug-in!Internal|bfit}\index{Plug-in!External|bfit}
Internal plug-ins are those distributed within the \FramaC kernel while
external plug-ins are those distributed independently of the \FramaC
kernel. They only differ in the way they are installed:
internal plug-ins are automatically installed with the \FramaC kernel,
while external plug-ins must be installed separately.

\section{Installing External Plug-ins}\label{sec:install-external}
\index{Plug-in!External}

To install an external plug-in, \FramaC itself must be properly installed
first. In particular, \texttt{frama-c \optionuse{-}{print-share-path}}
must return the share directory of \FramaC,
while \texttt{frama-c \optionuse{-}{print-lib-path}}
must return the directory where the \FramaC compiled library is installed.

The standard way for installing an external plug-in from source is to run the
sequence of commands \texttt{make \&\& make install}. Please refer to each
plug-in's documentation for installation instructions.

\section{Loading Plug-ins}\label{sec:use-plugins}

At launch, \FramaC loads all plug-ins in the
directories indicated by \texttt{frama-c \optionuse{-}{print-plugin-path}}.
These directories contain \texttt{META} files which are used by Dune to
automatically load these plug-ins.

Like other OCaml libraries, \FramaC plug-ins can be located via the environment
variable \texttt{OCAMLPATH}\index{OCAMLPATH}. It does not need to be set by
default, but if you install \FramaC in a non-standard directory
(e.g. \texttt{<PREFIX>}), you may need to add directory \texttt{<PREFIX>/lib}
to \texttt{OCAMLPATH}.

To prevent \FramaC from automatically loading any plug-ins, you can use option
\optiondef{-}{no-autoload-plugins}. Then the plugin to load can be selected
using \texttt{\optiondef{-}{load-plugin} <plugin-name>} (e.g. \texttt{aorai}).
Since \FramaC plugins are also OCaml libraries it is possible to use
\texttt{\optiondef{-}{load-library} <library-name>}
(e.g. \texttt{frama-c-aorai}).
Both options accept comma-separated lists of names.

\begin{important}
In general, plug-ins must be compiled with the
very same \caml compiler than \FramaC was, and against a consistent \FramaC
installation. Loading will fail and a warning will be emitted at launch if this
is not the case.
\end{important}

\subsection{Loading Single OCaml Files as Plug-ins}

\FramaC used to have an option \texttt{-load-script} that allowed loading a
single OCaml file as a mini-plug-in. Since \FramaC 26 (Iron), the use of Dune
requires a different approach: you need to create a directory containing the
following files:
\begin{itemize}
\item a \texttt{dune-project} file;
\item a \texttt{dune} file;
\item the \texttt{script.ml} file that you want to load with \FramaC.
\end{itemize}
With these files, you will be able to run \texttt{dune build} to compile the
script, and then \texttt{frama-c -load-module script.cmxs} to load it.

Here is an example \texttt{dune-project} file:
\begin{dunecode}
(lang dune 3.13)
\end{dunecode}

Note: you can match the language version (here, 3.7) to the one of your
installed \texttt{dune} package. Later versions often enable additional
warnings.

Here is an example \texttt{dune} file:
\begin{dunecode}
(executable
  (name "script") ; must match the name of the .ml file
  (modes plugin)
  (libraries frama-c.init.cmdline frama-c.kernel) ; add more if needed
  (flags -open Frama_c_kernel :standard)
  (promote (until-clean)) ; keeps script.cmxs in the base directory
)
\end{dunecode}

If your script depends on \FramaC plug-ins, you need to add them to
\texttt{libraries}, e.g. \texttt{frama-c-<plugin>.core}.

% Local Variables:
% ispell-local-dictionary: "english"
% TeX-master: "userman.tex"
% compile-command: "make"
% End:
