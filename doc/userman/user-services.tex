\chapter{General Kernel Services}\label{user-services}

This chapter presents some important services offered by the \FramaC platform.

\section{Projects}\label{sec:project}\index{Project|bfit}

A \FramaC project groups together one source code with the states (parameters,
results, \etc) of the \FramaC kernel and analyzers.

In one \FramaC session, several projects may exist at the same time, while
there is always one and only one so-called \emph{current} project in which
analyses are performed.  Thus projects help to structure a code analysis
session into well-defined entities. For instance, it is possible to perform an
analysis on the same code with different parameters and to compare the obtained
results. It is also possible to extract a program $p'$ from an initial program
$p$ and to compare the results of an analysis run separately on $p$ and $p'$.

\subsection{Creating Projects}

A new project is created in the following cases:
\begin{itemize}
\item at initialization time, a default project is created; or
\item \emph{via} an explicit user action in the GUI; or
\item a source code transforming analysis has been made. The analyzer then
  creates a new project based on the original project and containing the
  modified source code. A typical example is code slicing which tries to
  simplify a program by preserving a specified behaviour.
\end{itemize}

\subsection{Using Projects}

The list of existing projects of a given session is visible in the graphical
mode through the \texttt{Project} menu (see Section~\ref{sec:menubar}). Among
other actions on projects (duplicating, renaming, removing, saving, \etc), this
menu allows the user to switch between different projects during the same
session.

In batch mode, the only way to handle a multi-project session is through
the command line options \optionuse{-}{then-on}, \optionuse{-}{then-last} or
\optionuse{-}{then-replace} (see Section~\ref{sec:then}). It is also possible to
remove existing projects through the option \optiondef{-}{remove-projects}. It
might be useful to prevent prohibitive memory consumptions. In particular, the
category \texttt{@all\_but\_current} removes all the existing projects, but the
current one.

\subsection{Saving and Loading Projects}\label{sec:saveload}

A session can be saved to disk and reloaded by using the options
\texttt{\optiondef{-}{save} <file>} and \texttt{\optiondef{-}{load} <file>}
respectively. Saving is performed when \FramaC exits without error. In case of a
fatal error or an unexpected error, saving is done as well, but the generated
file is modified into \texttt{file.crash} since it may have been corrupted. In
other error cases, no saving is done. The same operations are available
through the GUI.

When saving, \emph{all} existing projects are dumped into an unique
non-human-readable file.

When loading, the following actions are done in sequence:
\begin{enumerate}
\item all the existing projects of the current session are deleted;
\item all the projects stored in the file are loaded;
\item the saved current project is restored;
\item \FramaC is replayed with the parameters of the saved current project,
  except for those parameters explicitly set in the current session.
\end{enumerate}

Consider for instance the following command.
\begin{frama-c-commands}
\$ frama-c -load foo.sav -eva
\end{frama-c-commands}
It loads all projects saved in the file \texttt{foo.sav}. Then, it runs the
value analysis in the new current project if and only if it was not already
computed at save time.

\begin{convention}
Saving the result of a time-consuming analysis before trying to use it in
different settings is usually a good idea.
\end{convention}

\begin{important}
Beware that all the existing projects are deleted, even if an error occurs when
reading the file. We strongly recommend you to save the existing projects before
loading another project file.
\end{important}

\paragraph{Special Cases}

Options \optionuse{-}{help}, \optionuse{-}{verbose},
\optionuse{-}{debug}\xspace(and
their corresponding plugin-specific counterpart)
as well as \optionuse{-}{explain}, \optionuse{-}{quiet}\xspace and
\optionuse{-}{unicode}\xspace are not saved on disk.

\section{Dependencies between Analyses}

Usually analyses do have parameters (see Chapter~\ref{user-analysis}). Whenever
the values of these parameters change, the results of the analyses may also
change. In order to avoid displaying results that are inconsistent with the
current value of parameters, \FramaC automatically discards results of an
analysis when one of the analysis parameters changes.

Consider the two following
commands.\optionidx{-}{save}\optionidx{-}{load}\optionidx{-}{absolute-valid-range}%
\optionidx{-}{ulevel}
\begin{frama-c-commands}
\$ frama-c -save foo.sav -ulevel 5 -absolute-valid-range 0-0x1000 -eva foo.c
\$ frama-c -load foo.sav
\end{frama-c-commands}
\FramaC runs the value analysis plug-in on the file \texttt{foo.c} where loops
are unrolled $5$ times (option \texttt{-ulevel}, see
Section~\ref{sec:normalize}). To compute its result, the value analysis
assumes the memory range 0:0x1000 is addressable  (option
\texttt{-absolute-valid-range}, see Section~\ref{sec:customizing-analyzers}).
Just
after, \FramaC saves the results on file \texttt{foo.sav} and exits.

At loading time, \FramaC knows that it is not necessary to redo the value
analysis since the parameters have not been changed.

Consider now the two following commands.
\optionidx{-}{save}\optionidx{-}{load}\optionidx{-}{absolute-valid-range}%
\optionidx{-}{ulevel}
\begin{frama-c-commands}
\$ frama-c -save foo.sav -ulevel 5 -absolute-valid-range 0-0x1000 -eva foo.c
\$ frama-c -load foo.sav -absolute-valid-range 0-0x2000
\end{frama-c-commands}
The first command produces the very same result than above. However, in the
second (load) command, \FramaC knows that one parameter has changed. Thus it
discards the saved results of the value analysis and recomputes it on the same
source code by using the parameters
\texttt{-ulevel 5 -absolute-valid-range 0-0x2000} (and the
default value of each other parameter).

In the same fashion, results from an analysis $A_1$ may well depend on results from another
analysis $A_2$. Whenever the results from $A_2$ change, \FramaC automatically discards
results from $A_1$. For instance, slicing results depend on value analysis
results; thus the slicing results are discarded whenever the value analysis
ones are.
