\chapter{Conclusion}

En conclusion, on peut dire que les fonctionnalités de base de l'outil
n'ont pas beaucoup évoluées en 2008, mais il a gagné en robustesse,
et en précision~: ce qui était le principal objectif de l'année.\\

La principale évolution concerne la gestion des annotations
que ce soit lors de la production du résultat (que garde-t-on~?)
que comme critère de \slicing. Ce point est encore en développement
car il nécessite l'utilisation de fonctions externes au module
qui n'existent pas encore.\\

La boîte à outils de \slicing peut être considéré comme stable,
même si elle peut encore évoluer pour répondre à de nouveaux besoins,

L'annexe \ref{sec-projets} présente en particulier
certains projets qui ont été évoqués,
et dont certain pourrait éventuellement venir compléter l'outil.


