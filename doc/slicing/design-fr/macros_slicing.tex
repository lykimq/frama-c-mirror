%==============================================================================
\usepackage{tabularx}
\usepackage{alltt}
\usepackage{ulem} %sout

\usepackage{amsmath}
\usepackage{wasysym}
\usepackage{stmaryrd}
\usepackage{listings}
\usepackage{makeidx}

\usepackage[french]{varioref}

%------------------------------------------------------------------------------
\newcommand{\setenvclass}[2]{}

\usepackage{listings}
% TODO :
\lstdefinestyle{clststyle}{}
\lstnewenvironment{clisting}
  {
   \setenvclass{lstlisting}{clisting}
   \lstset{language=C,
           style=clststyle,
           inputencoding=utf8,
           extendedchars=true,
           literate={ç}{{\c{c}}}1
                    {â}{{\^{a}}}1
                    {à}{{\`{a}}}1
                    {ê}{{\^{e}}}1
                    {ë}{{\"{e}}}1
                    {é}{{\'{e}}}1
                    {è}{{\`{e}}}1
                    {ï}{{\"{\i}}}1
                    {î}{{\^{\i}}}1
                    {ô}{{\^{o}}}1
                    {û}{{\^{u}}}1
                    {ü}{{\"{u}}}1}
                    }
  {}

\newcommand{\camlListingInput}[1]{
  \begin{footnotesize}
  \lstinputlisting[language=Ocaml,
                  style=ocaml-basic,
                  inputencoding=utf8,
                  extendedchars=true,
                  literate={ç}{{\c{c}}}1
                           {â}{{\^{a}}}1
                           {à}{{\`{a}}}1
                           {ê}{{\^{e}}}1
                           {ë}{{\"{e}}}1
                           {é}{{\'{e}}}1
                           {è}{{\`{e}}}1
                           {ï}{{\"{\i}}}1
                           {î}{{\^{\i}}}1
                           {ô}{{\^{o}}}1
                           {û}{{\^{u}}}1
                           {ü}{{\"{u}}}1]{#1}
  \end{footnotesize}
}

%-----------------------------------------------
% Commandes pour mettre des ref dans l'index :
\newcommand{\idb}[1]{\textbf{#1}}
\newcommand{\indextxt}[1]{\index{#1}{\bf #1}}
\newcommand{\indexdef}[1]{\index{#1|idb}{{\bf #1}}}
\newcommand{\indextxtdef}[2]{\index{#2|idb}{{\bf #1}}}

%-----------------------------------------------
% quelques "mots" spéciaux
\newcommand{\slicing}{{\it slicing}\xspace}
\newcommand{\slicingb}{{\it slicing\ }}
           % permet de corriger les cas où \slicing est suivi d'une {
           % car il semble que xspace gère mal ce cas.
\newcommand{\caml}{{\sc Ocaml}\xspace}
\newcommand{\caveat}{{\sc Caveat}\xspace}
\newcommand{\cil}{{\sc CIL}\xspace}
\newcommand{\ppc}{{\sc Frama-C}\xspace}
\newcommand{\from}{{\sc From}\xspace}

\newcommand{\mylet}{\text{\it let }}
\newcommand{\myin}{\text{\it in }}

\DeclareMathOperator{\Call}{\mathcal{C\mathit{all}}}

\makeatletter
\newcommand*{\motcle}[3]{%
  \index{#2!#1}%
  \expandafter \@ifdefinable \csname #3#1\endcsname {%
    \global \@namedef{#3#1}{\textit{#1}}%
    }%
}
\makeatother

%------------------------------------------------------------------------------
\newcommand{\verbtt}[1]{{\small{\tt{#1}}}}
%------------------------------------------------------------------------------
% Légende des figures :
%\renewcommand{\captionlabeldelim}{ : }
%-----------------------------------------------
% prend le nom du fichier
% et éventuellement un facteur d'échelle (1 par défaut)
\newcommand{\uneimage}[1]{
    \includegraphics[width=1\textwidth]{#1}
}
%--------------------------------
\newcommand{\labcenterepsfig}[3]{
  \begin{center}
  \begin{figure}[hbt]
    \centering{
    \leavevmode
    \uneimage{#1}
    }
    \caption{\label{#3} #2}
  \end{figure}
  \end{center}
  }
%------------------------------------------------------------------------------
\newenvironment{debutenv}[1]
  {
    \vspace{5mm}
    \begin{footnotesize}
      \setenvclass{minipage}{myminipage}
      \noindent\begin{minipage}[t]{\textwidth}
      \fbox{\bf #1} \hrule
    \vspace{3mm}
  }
  {
    \end{minipage}
    \end{footnotesize}
  }
\newenvironment{finenv}
  {
    \begin{footnotesize}
      \setenvclass{minipage}{myminipage}
    \noindent\begin{minipage}[t]{\textwidth}
  }
  {
    \hrule
    \
    \end{minipage}
    \end{footnotesize}
    \vspace{5mm}
  }
\newenvironment{monenv}[1]
  {\begin{debutenv}{#1}}
  {\end{debutenv}
   \par
   \begin{finenv}
   \end{finenv}}

\newcounter{numexemple}
\newcommand{\titreexemple}{%
  \stepcounter{numexemple}{Exemple \arabic{numexemple}}}

\newenvironment{exemple}
  {\begin{monenv}{\titreexemple}}
  {\end{monenv}}

\newenvironment{exemple1}
  {\begin{debutenv}{\titreexemple}}
    {\end{debutenv}}

\newenvironment{exemple2}
  {\begin{finenv}}
    {\end{finenv}}

\newenvironment{astuce}
  {\begin{monenv}{Astuce}}
  {\end{monenv}}

\newenvironment{definition}[1]
  {\begin{monenv}{Definition (#1)}}
  {\end{monenv}}

\newenvironment{algo}[1]
  {\begin{monenv}{Algorithme (#1)}}
  {\end{monenv}}

\newcounter{numrque}
\newenvironment{remarque}[1]
  {\refstepcounter{numrque}{\bf Remarque \arabic{numrque}} :\label{#1}}
  {}
%==============================================================================
