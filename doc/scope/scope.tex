\documentclass[a4paper,twoside]{article}

\usepackage[utf8]{inputenc}
\usepackage[T1]{fontenc}
\usepackage[french]{babel}

\usepackage{url}
\usepackage{hevea}
\begin{htmlonly}
\loadcssfile{../hevea.css}
\loadcssfile{local.css}
\end{htmlonly}
\begin{latexonly}
\renewcommand{\cite}[1]{\citep{#1}}
\end{latexonly}

\newcommand{\latexurl}[1]{\url{#1}}
\renewcommand{\url}[1]{\ahrefurl{#1}}

%------------------------------------
\renewcommand{\sc}{\mathcal{S}}
\newcommand{\sca}{\mathcal{S}_a}
\newcommand{\scb}{\mathcal{S}_b}
%------------------------------------
\newcommand{\letitre}{Portée d'une donnée et d'une propriété}
\title{\letitre}
\author{Anne Pacalet}
\newcommand{\ladate}{\today}

% \usepackage{fancyheadings} obsolete : replace by \usepackage{fancyhdr}
\usepackage{fancyhdr}
%------------------------------------

%==============================================================================
\begin{document}

\maketitle
\begin{htmlonly}
\begin{center}
Existe aussi au format
\ahref{spec.ps}{postscript}
ou
\ahref{spec.pdf}{pdf}.
\end{center}
\end{htmlonly}

% Table des matières
%\tableofcontents
%\clearpage

% entête et pied de page
\pagestyle{fancyplain}
\lhead{\small{\letitre}}
\rhead{\small{\ladate}}
\chead{}
\cfoot{\thepage}

%==============================================================================

\section{Introduction}

On utilise souvent le terme de portée ({\it scope} en anglais)
en parlant d'une variable pour désigner la zone du programme
dans laquelle il est valide de l'utiliser.
Nous parlerons ici de la {\bf portée d'une donnée} pour désigner
de l'ensemble des points de programme pour lesquels la donnée n'est pas
modifiée. Nous préciserons en \S\ref{data-scope} ce que cela veut dire.\\

Puis, nous généraliserons en \S\ref{prop-scope} à la
{\bf portée d'une propriété}.

\section{Portée d'une donnée}\label{data-scope}

\subsection{Donnée vs. {\it lvalue}}

Nous utilisons ici le terme de {\bf donnée} pour représenter une zone mémoire.
Une donnée est généralement désignée
par une expression de la famille des {\it lvalues} qui caractérise
les expressions qui apparaissent à gauche d'une affectation.
Lorsqu'on parle de la donnée correspondant à une variable X,
il n'y a pas d'ambiguïté, il s'agit précisement la zone mémoire allouée pour
ranger la valeur de X.
En revanche, lorsque la donnée est désignée par {\tt T[i]},
la traduction en terme de zone mémoire fait intervenir la valeur de i.
La donnée est alors une sur-approximation de la zone mémoire
dont il faut connaître la valeur pour évaluer l'expression.
La donnée doit donc contenir la zone mémoire allouée pour i,
et une partie de la zone allouée pour T, éventuellement toute si on n'a aucune
information sur i.

\subsection{Discussion sur la portée}

On souhaite déterminer l'ensemble $\sc(D@L)$ des points
pour lesquels la donnée D a la même valeur qu'au point L.
Cette définition semble relativement claire au premier abord,
mais nous allons voir qu'elle manque quelque peu de précision...\\

Pour parler de la valeur d'une donnée en différents points,
nous allons raisonner en terme de traces d'exécution.
Si on cherche à déterminer $\sc(D@L)$,
il peut sembler naturel de ne s'intéresser qu'aux traces qui passent par L
car que signifierait {\it avoir la même valeur qu'en L} pour les autres traces~?

Mais qu'en est-il des points appartenant à $\sc$ ?

Pour la séquence~:

\centerline{\verb!L: if (c) L1:.. else L2:...!}

les points L1 et L2 n'apparaîtront pas dans une même trace, et pourtant,
on aimerait bien qu'ils appartiennent tous les deux à $\sc(D@L)$.

Re-formulons donc : si $P \in \sc$, on veut que pour toutes les traces qui passent
par L et P, D ait la même valeur.
Mais avec cette définition, si on a :

\begin{verbatim}
L0 :...
if (c) {
  L1:..
  }
else {
  L2a:...
  D=x;
  L2b:...
}
L3:...
\end{verbatim}

on va avoir $L3 \in \sc(D@L1)$. Est-ce que c'est bien ce qu'on veut avoir ?
Comme par ailleurs, on a évidemment $L0 \in \sc(D@L1)$, est-ce qu'on aura pas
tendance à penser que D à la même valeur en L0 et en L3 ?
Il faudra mettre en évidence le fait qu'on s'intéresse uniquement
aux traces passant pas L1...\\

Au premier abord, on a l'intuition qu'une relation
{\it avoir la même valeur pour D} est une relation d'équivalence, mais ce n'est
si simple étant donné qu'on
ne considère pas les mêmes traces. Si c'était le cas, on se retrouverait
rapidement à dire que D a la même valeur en L2a et L2b en propageant en
arrière à partir de L3, et en avant à partir de L0 !!!

Par ailleurs, qu'est-ce que ça veut dire pour les traces
qui contiennent plusieurs fois L et/ou P ?

Prenons quelques exemples :

\begin{itemize}
  \item si on a \verb!L:... P:... D=x;... P:...!
     ou \verb!P:... D=x;... P:... L:...!
     \begin{itemize}\item[$\rightarrow$]
   $P \notin \sc(D@L)$
   car il y a des chemins de L à P ou de P à L, pour lesquels D est modifiée.
     \end{itemize}

   \item si on a une boucle \verb!Lp:... L:... n1:... D=x; n2:... goto Lp;!
     \begin{itemize}\item[$\rightarrow$]
	   quand on va en avant de L à n1, D n'est pas modifié,
	   mais elle l'est quand on va de n1 à L, et inversement pour n2.
     \end{itemize}
\end{itemize}

Il semble alors judicieux de distinguer une sélection avant $\sca$
et une sélection arrière $\scb$, pour pouvoir avoir $n1 \in \sca$ et $n2 \in \scb$.
La signification serait alors :
$P \in \sca$ ssi à chaque fois que l'on va de L à P,
D a la même valeur en ces deux points,
mais il faut se restreindre aux chemins qui ne font pas un tour de boucles...
càd ceux qui ne repassent pas par L.

Mais si on a :

\centerline{\verb!P:... Lp:... L:... D=x; ... goto Lp;!}

quel est l'argument pour exclure P de $\scb$ ?
On a bien : pour tous les chemins qui vont de P à L sans repasser par L,
  D a la même valeur ! alors quoi ???

En avant, on a le même problème avec :

\centerline{\verb!Lp:... D=x; ... L:... goto Lp; P:...!}

Pourquoi est-ce qu'on n'aurait pas $P \in \sca(D@L)$ ?\\

Ou alors, on dit que c'est normal~?

\subsection{Définitions}

On définit finalement les ensembles en considérant les traces d'exécution
qui traversent une fonction
sans rentrer dans les appels de fonction (en cas d'appel récursif, on ne
considère donc qu'une exécution de la fonction).
On considère les exécutions de la fonction
qui passent par L, et on observe la valeur de la donnée en L et en P
comme si on en regardait la trace dans un débogueur.


\begin{itemize}
  \item $P \in \sca(D@L)$ si et seulement pour toute exécution de la fonction
    qui passe par L, si on s'arrête en P,
    D a toujours la même valeur
    que la {\bf dernière} fois que l'on est passé en L~;
  \item $P \in \scb(D@L)$ si et seulement si pour toute exécution
    de la fonction qui passe par L, si on s'arrête en P,
    D a toujours la même valeur que
    la {\bf prochaine} fois que l'on passe en L.
\end{itemize}

\subsection{Algorithmes}

En pratique, les résultats sont présentés comme une liste d'instructions et non
directement de points de programme. Il faut donc préciser
qu'on choisit de considérer l'étiquette de départ L comme une instruction
à part entière, et l'instruction $I_L$ comme étant l'instruction suivante~:
donc en suivant la définition, si on place un point d'arrêt en $I_L$,
D a toujours la même valeur que la dernière fois que l'on est passé en L,
et on a donc toujours $I_L \in \sca(D,L)$.\\

On calcule tout d'abord pour chaque donnée modifiée par la fonction,
une liste des instructions qui la modifie.\\

Puis, on fait une analyse {\it dataflow} en avant et une autre en arrière
dans lesquelles on associe une marque à chaque point de la fonction parmi :

\begin{itemize}
  \item {\tt Start}~: point de départ de l'analyse (marque du point L)~;
  \item {\tt NotSeen}~: point non rencontré~;
  \item {\tt Modif}~: donnée modifiée~;
  \item {\tt SameVal}~: donnée non modifiée.\\
\end{itemize}

L'opération qui permet d'accumuler la valeur en un point, que ça soit lors de
l'analyse avant ou l'analyse arrière, est~:
\begin{verbatim}
let merge b1 b2 = match b1, b2 with
    | Start, _ | _, Start -> Start
    | NotSeen, b | b, NotSeen -> b
    | Modif, _ | _, Modif -> Modif
    | SameVal, SameVal -> SameVal
\end{verbatim}

On se donne aussi une fonction de transfert $T(I,m)$ qui vaut~:
\begin{itemize}
  \item {\tt Modif} si l'instruction I modifie la donnée qui nous intéresse,
  \item {\tt SameVal} si m={\tt Start}
  \item m sinon.
\end{itemize}

\subsubsection{Analyse avant}

Il s'agit de calculer une marque $m_a(P)$ pour chaque point P.

On commence en initialisant le point de départ $m_a(L)$ à {\tt Start}.
Puis on lance le calcul en indiquant qu'il faut traiter l'instruction L.

Pour chaque chaque instruction I à traiter, le calcul consiste à
calculer la marque $m' = T(I, m_a(I))$ à propager à ses successeurs.
Pour chaque successeur I' de I, on va ajoute la marque m'
à un éventuel résultat précédent $m_a(I')$
à l'aide de l'opération {\tt merge}
($m_a(I')=m'$ si on n'a encore rien calculé).
Si $m_a(I')$ n'est pas modifiée par l'opération, on arrête de propager par ce
chemin. Sinon, on met I' dans la liste des instructions à traiter.

Quand le résultat est stabilisé (plus d'instructions à traiter),
on a donc~:
$$
m_a(I') = \bigcup_{I \in pred(I')} T(I, m_a(I))
$$

On récupère la liste des instructions
marquées {\tt Start} ou {\tt SameVal}, et cela nous donne $\sca(D,L)$.\\

Pour tout $P \in \sca(D,L)$, si P n'est pas L, on a donc~:
$$
\bigcup_{I \in pred(P)} T(I, m_a(I)) = {\tt SameVal}
$$
c'est-à-dire que dans les prédécesseurs de $P$ on ne peut avoir que des
instructions I ne modifiant pas D et ayant une marque
{\tt Start} ou {\tt SameVal}, c'est-à-dire des éléments de $\sca(D,L)$
ou bien une marque {\tt NotSeen} c'est-à-dire une instruction que l'on ne peut
pas atteindre depuis L.

On peut remarquer que l'on n'a pas d'information sur les prédécesseurs de L,
car quelque soient leurs marques, la marque de L reste à  {\tt Start}.

\subsubsection{Analyse arrière}

L'analyse est similaire à l'analyse avant, sauf qu'elle s'effectue en arrière~!
Il s'agit encore de calculer une marque $m_b(P)$ pour chaque point P.

Cela nécessite d'initialiser tous les points à {\tt NotSeen},
sauf le point de départ L qui est initialisé à  {\tt Start}.
Puis on lance le calcul en indiquant qu'il faut traiter
les prédécesseurs de L.

Pour chaque chaque instruction I à traiter, le calcul consiste à
calculer la marque m' en combinant
les marques associées aux successeurs de I à l'aide de l'opération
{\tt merge}.
On calcule ensuite $m = T(I, m')$.
On remarque que c'est la même fonction de transfert qu'en avant.

Si $m_b(I) = m_b(I) \cup m$, on arrête la propagation,
et sinon, on ajoute m à $m_b(I)$
et on ajoute les prédécesseurs de I dans les  instructions à traiter.\\

Quand le résultat est stabilisé, on a donc~:
$$
m_b(I) = T(I, \bigcup_{I' \in succ(I)} m_b(I'))
$$
On récupère la liste des instructions
marquées {\tt SameVal},
et on ajoute l'instruction L de départ marquée {\tt Start}
si $T(L, \bigcup_{I \in succ(L)} m_b(I))$ vaut {\tt SameVal}.
Cela nous donne $\scb(D,L)$.\\

Pour tout $P \in \scb(D,L)$, on a donc~:
$$
T(P, \bigcup_{I \in succ(P)} m_b(I)) = {\tt SameVal}
$$
c'est-à-dire que P ne modifie pas D, et que pour tout successeur I de P,
soit il s'agit du point de départ L, soit I est aussi dans $\scb(D,L)$.

\subsection{Propriétés}



Pour tout $P \in \sca(D@L)$, il existe nécessairement un chemin entre L et P,
et tous les points du chemin sont aussi dans $ \sca(D@L)$.
De même, pour tout $P \in \scb(D@L)$,
il existe nécessairement un chemin entre P et L,

Il n'y a pas forcement de relation de domination/postdomination entre P et L.\\

Relation entre $\sc(D@L)$ et $\sc(D@P)$ quand $P \in \sc(D@L)$~?...

\section{Portée d'une propriété}\label{prop-scope}

\subsection{Problème général}

On s'intéresse maintenant au déplacement d'une propriété dans une fonction.
On pourrait penser qu'il suffit de déterminer les données qui servent à évaluer
la propriété et de calculer la portée de ces données pour pouvoir déplacer la
propriété en un point appartenant à la portée de ces données.
Mais ce n'est pas si simple car  si on considère la séquence~:

\centerline{\verb!L1:...  if (c) \{ L2:... \} else \{ L3a: x = 2;
L3b:... \} L4:...!}


\begin{itemize}
  \item si on a une propriété P portant sur x en L2 (disons $x>0$), on ne peut pas
    forcement la déplacer en L1 car elle peut très bien n'être vraie que dans le
    cas où la condition c est vraie (par exemple, grâce à une précondition :
    $c \Rightarrow x>0$);
  \item par contre, si on a P en L1, elle est aussi vraie en L2, mais il n'y a pas équivalence, car il faut aussi qu'elle soit vraie en L3a.
  \item idem dans l'autre sens :
    \begin{itemize}
      \item $P@L4 \Rightarrow P@L3b$
      \item mais on n'a pas~: $P@L3b \Rightarrow P@L4$
    \end{itemize}
\end{itemize}

\subsection{Menaces redondantes}

Dans un premier temps, pour allons nous intéresser aux propriétés générées
pour exprimer les menaces. Il s'agit d'assertions qui expriment les conditions
qu'il faut vérifier pour que les instructions ne {\it plantent} pas.
Si on a un accès à un pointeur p par exemple, et que l'analyse de valeur
détecte que ce pointeur ne contient pas toujours une adresse valide,
on va voir apparaître~:

\begin{verbatim}
/*@ assert \valid(p);
      // synthesized
 */
x = *p;
\end{verbatim}

L'analyse de valeur considère dans la suite des calculs que cette propriété
est vérifiée. Néanmoins, dans certains cas, elle n'est pas suffisamment précise
pour retenir cette information, et la menace est donc émise plusieurs fois, à
chaque accès au pointeur. Par exemple, si on ne sait rien sur p,
on a~:
\begin{verbatim}
/*@ assert \valid(p);
        // synthesized
 */
   x = *p;
 /*@ assert \valid(p);
       // synthesized
 */
  *p = 3;
\end{verbatim}

\subsection{Réduction des menaces}

On travaille uniquement sur la portée des données,
c'est-à-dire sur le déplacement de propriétés à travers des instructions
qui ne modifient pas les données utilisées.
La comparaison des menaces sera donc uniquement syntaxique.\\

Si une propriété P est vraie à un point de programme L1,
elle est aussi vraie pour les points L2 qui suivent L1 tels que~:
\begin{itemize}
  \item L2 est dominé par L1
(ie. si on passe en L2, on est forcement passé en L1)
\item et la propriété n'est pas modifiée entre L1 et L2.
\end{itemize}
Donc si on rencontre une menace identique à P en un point L2
qui vérifie ces caractéristiques, elle peut être éliminée.\\

Si une propriété est vraie à un point de programme L1,
on pourrait aussi se dire qu'elle est
vraie pour les points L0 qui précèdent L1 tels que~:
\begin{itemize}
  \item L1 postdomine L0
    (ie. si on passe en L0, on passe forcement en L1)
  \item  et la propriété n'est pas modifiée entre L0 et L1.
\end{itemize}
Mais le déplacement d'une propriété en arrière est plus délicate,
car à cause des problèmes de terminaison des instructions que l'on traverse,
ça peut conduire à la rendre plus forte que nécessaire.


%------------------------------------------------------------------------------
\end{document}
%==============================================================================
