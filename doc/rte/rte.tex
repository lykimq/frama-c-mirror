\newcommand{\bropen}{\mbox{\tt [}}
\newcommand{\brclose}{\mbox{\tt ]}}
\newcommand{\cbopen}{\mbox{\tt \{}}
\newcommand{\cbclose}{\mbox{\tt \}}}
\newcommand{\cnn}{\mbox{ISO C99}}
\newcommand{\optsigned}{\mbox{\lstinline|-rte-signed|}}
\newcommand{\optnodowncast}{\mbox{\lstinline|-rte-no-downcast|}}
\newcommand{\rte}{\textsf{RTE}\xspace}
\newcommand{\framac}{\textsf{Frama-C}\xspace}
\newcommand{\acsl}{\textsf{ACSL}\xspace}
\newcommand{\gcc}{\textsf{Gcc}\xspace}
\newcommand{\clang}{\textsf{Clang}\xspace}

\tableofcontents

\chapter{Introduction}\label{introduction}
\vspace{1cm}

\section{\rte{} plug-in}

This document is a reference manual for the annotation generator plug-in called
\rte{}.  The aim of the \rte{} plug-in is to automatically generate annotations
for:

\begin{itemize}
\item common runtime errors, such as division by zero, signed integer overflow
  or invalid memory access;
\item unsigned integer overflows, which are allowed by the C language but may
  pose problem to solvers;
\end{itemize}

In a modular proof setting, the main purpose of the \rte{} plug-in is to seed
more advanced plug-ins (such as the weakest-preconditions generation
plug-in~\cite{framacwp}) with proof obligations.  Annotations can also be
generated for their own sake in order to guard against runtime errors.  The
reader should be aware that discharging such annotations is much more difficult
than simply generating them, and that there is no guarantee that a plug-in such
as Frama-C's value analysis~\cite{framacvalueanalysis} will be able to do so
automatically in all cases.

\rte{} performs syntactic constant folding in order not to generate trivially
valid annotations.  Constant folding is also used to directly flag some
annotations with an invalid status.  \rte{} does not perform any kind of
advanced value analysis, and does not stop annotation generation when flagging
an annotation as invalid, although it may generate fewer annotations in this
case for a given statement.

Like most Frama-C plug-ins, \rte{} makes use of the hypothesis that signed
integers have a two's complement representation, which is a common
implementation choice.  Also note that annotations are dependent of the {\it
  machine dependency} used on Frama-C command-line, especially the size of
integer types.

The C language ISO standard \cite{standardc99} will be referred to as \cnn{}
(of which specific paragraphs are cited, such as \mbox{6.2.5.9}).

%%\section{Generated Annotations}

\section{Runtime errors}

A runtime error is a usually fatal problem encountered when a program is
executed.  Typical fatal problems are segmentation faults (the program tries to
access memory that it is not allowed to access) and floating point exceptions
(for instance when dividing an integer by zero: despite its name, this exception
does not only occur when dealing with floating point arithmetic).  A C program
may contain ``dangerous'' constructs which under certain conditions lead to
runtime errors when executed.  For instance evaluation of the expression
\lstinline|u / v| will always produce a floating point exception when
\lstinline|v = 0| holds.  Writing to an out-of-bound index of an array may
result in a segmentation fault, and it is dangerous even if it fails to do so
(other variables may be overwritten).  The goal of this Frama-C plug-in is to
detect a number of such constructs, and to insert a corresponding logical
annotation (a first-order property over the variables of the construct) ensuring
that, whenever this annotation is satisfied before execution of the statement
containing the construct, the potential runtime error associated with the
expression will not happen.  Annotation checking can be performed (at least
partially) by Frama-C value analysis plug-in~\cite{framacvalueanalysis}, while
more complicated properties may involve other plug-ins and more user
interaction.

At this point it is necessary to define what one means by a ``dangerous''
construct.  \cnn{} lists a number of {\it undefined} behaviors (the program
construct can, at least in certain cases, be erroneous), a number of {\it
  unspecified} behaviors (the program construct can be interpreted in at least
two ways), and a list of {\it implementation-defined} behaviors (different
compilers and architectures implement different behaviors).  Constructs leading
to such behaviors are considered dangerous, even if they do not systematically
lead to runtime errors.  In fact an undefined behavior must be considered as
potentially leading to a runtime error, while unspecified and
implementation-defined behaviors will most likely result in portability
problems. %%We will mainly focus on undefined behaviors, and thus on runtime
error prevention.

An example of an undefined behavior (for the C language) is {\it signed integer
  overflow}, which occurs when the (exact) result of a signed integer arithmetic
expression can not be represented in the domain of the type of the
expressions. For instance, supposing that an \lstinline|int| is 32-bits wide,
and thus has domain \lstinline|[-2147483648,2147483647]|, and that \lstinline|x|
is an \lstinline|int|, the expression \lstinline|x+1| performs a signed integer
overflow, and therefore has an undefined behavior, if and only if \lstinline|x|
equals \lstinline|2147483647|.  This is independent of the fact that for most
(if not all) C compilers and 32-bits architectures, one will get
\lstinline|x+1 = -2147483648| and no runtime error will happen.  But by strictly
conforming to
the C standard, one cannot assert that the C compiler will not in fact generate
code provoking a runtime error in this case, since it is allowed to do so.
%% In fact, for an expression such as \lstinline|x/y| (for \lstinline|int x,y|),
%% the execution will most likely result in a floating point exception
%% when \lstinline|x = -2147483648, y = -1| (the result is \lstinline|2147483648|, which overflows).
Also note that from a security analysis point of view, an undefined behavior
leading to a runtime error classifies as a denial of service (since the program
terminates), while a signed integer overflow may very well lead to buffer
overflows and execution of arbitrary code by an attacker.  Thus not getting a
runtime error on an undefined behavior is not necessarily a desirable behavior.

On the other hand, note that a number of behaviors classified as
implementation-defined by the ISO standard are quite painful to deal with in
full generality.  In particular, \cnn{} allows either {\it sign and magnitude},
{\it two's complement} or {\it one's complement} for representing signed integer
values.  Since most if not all ``modern'' architectures are based on a {\it
  two's complement} representation (and that compilers tend to use the hardware
at their disposal), it would be a waste of time not to build verification tools
by making such wide-ranging and easily checkable assumptions.  {\bf Therefore
  \rte{} uses the hypothesis that signed integers have a {\it two's complement}
  representation.}
%% value analysis makes the same assumption; also see value analyse manual 4.4.1

%% Frama-C is not intended to work on non ISO conforming inputs (?),
%% but conforming programs may still produce undefined behaviors. Well ...

\section{Other annotations generated}

\rte{} may also generate annotations that are not related to runtime errors:

\begin{itemize}

\item absence of unsigned overflows checking. Although unsigned overflows are
  well-defined, some plug-ins may wish to avoid them.

\item accesses to arrays that are embedded in a struct occur withing valid
  bounds. This is stricter than verifying that the accesses occurs within the
  struct.

\end{itemize}

%%%%%%%%%%%%%%%%%%%%%%%%%%%%%%%%%%%%%%%%%%%%%%%%%%%%%
%%%%%%%%%%%%%%%%%%%%%%%%%%%%%%%%%%%%%%%%%%%%%%%%%%%%%%
\chapter{Runtime error annotation generation}

\section{Integer operations}

According to \mbox{6.2.5.9}, operations on unsigned integers ``can never
overflow'' (as long as the result is defined, which excludes division by zero):
they are reduced modulo a value which is one greater than the largest value of
their unsigned integer type (typically $2^n$ for $n$-bit integers).  So in fact,
arithmetic operations on unsigned integers should really be understood as
modular arithmetic operations (the modulus being the largest value plus one).

On the other hand, an operation on {\em signed} integers might overflow and this
would produce an undefined behavior.  Hence, a signed integer operation is only
defined if its result (as a mathematical integer) falls into the interval of
values corresponding to its type (e.g. \lstinline|[INT_MIN,INT_MAX]| for
\lstinline|int| type, where the bounds \lstinline|INT_MIN| and
\lstinline|INT_MAX| are defined in the standard header \lstinline|limits.h|).
Therefore, signed arithmetic is true integer arithmetic as long as intermediate
results are within certain bounds, and becomes undefined as soon as a
computation falls outside the scope of representable values of its type.

The full list of arithmetic and logic operations which might overflow is
presented hereafter.  Most of these overflows produce undefined behaviors, but
some of them are implementation defined and indicated as such.

\subsection{Addition, subtraction, multiplication}

These arithmetic operations may not overflow when performed on signed operands,
in the sense that the result must fall in an interval which is given by the type
of the corresponding expression and the macro-values defined in the standard
header \lstinline|limits.h|.  A definition of this file can be found in the
\lstinline|share| directory of Frama-C.
%%%which is coherent with the bit-size of types specified with \lstinline|-machdep|.

\medskip

\begin{center}
\begin{tabular}{|l|l|}
\hline
type & representable interval \\
\hline
\lstinline|signed char| & \lstinline|[SCHAR_MIN, SCHAR_MAX]|  \\
\lstinline|signed short| & \lstinline|[SHRT_MIN,SHRT_MAX]| \\
\lstinline|signed int| & \lstinline|[INT_MIN,INT_MAX]| \\
\lstinline|signed long int| & \lstinline|[LONG_MIN,LONG_MAX]| \\
\lstinline|signed long long int| & \lstinline|[LLONG_MIN,LLONG_MAX]| \\
\hline
\end{tabular}
%%\caption{Signed integers: macros for min and max bounds}
\end{center}

\medskip

Since \rte{} makes the assumption that signed integers are represented in 2's
complement, the interval of representable values also corresponds to $[-2^{n-1},
  2^{n-1}-1]$ where $n$ is the number of bits used for the type (sign bit
included, but not the padding bits if there are any).  The size in bits of a
type is obtained through \lstinline|Cil.bitsSizeOf: typ -> int|, which bases
itself on the machine dependency option of Frama-C. For instance by using
\lstinline|-machdep x86_32|, we have the following (the size is expressed in
bits):
\begin{center}
\begin{tabular}{|l|c|l|}
\hline
type & size & representable interval \\
\hline
\lstinline|signed char| & 8 & \lstinline|[-128,127]|  \\
\lstinline|signed short| & 16 & \lstinline|[-32768,32767]| \\
\lstinline|signed int| & 32 & \lstinline|[-2147483648,2147483647]| \\
\lstinline|signed long int| & 32 & \lstinline|[-2147483648,2147483647]| \\
\lstinline|signed long long int| & 64 & \lstinline|[-9223372036854775808,9223372036854775807]| \\
\hline
\end{tabular}
%%\caption{Signed integer types: bit sizes and interval of values}
\end{center}

\medskip
Frama-C annotations added by plug-ins such as \rte{} may not contain macros
since preprocessing is supposed to take place beforehand (user annotations at
the source level can be taken into account by using the \lstinline|-pp-annot|
option).  As a consequence, annotations are displayed with big constants such as
those appearing in this table.

\begin{example} ~
Here is a \rte{}-like output in a program involving \lstinline|signed long int|
with an \lstinline|x86_32| machine dependency:
\begin{listing-nonumber}
int main(void) {
  signed long int lx, ly, lz;

  /*@ assert rte: signed_overflow: -2147483648 <= lx*ly; */
  /*@ assert rte: signed_overflow: lx*ly <= 2147483647; */
  lz = lx * ly;

  return 0;
}
\end{listing-nonumber}

The same program, but now annotated with an \lstinline|x86_64| machine
dependency (option \texttt{-machdep x86\_64}):
\begin{listing-nonumber}
int main(void) {
  signed long int lx, ly, lz;

  /*@ assert rte: signed_overflow: -9223372036854775808 <= lx*ly; */
  /*@ assert rte: signed_overflow: lx*ly <= 9223372036854775807; */
  lz = lx * ly;

  return 1;
}
\end{listing-nonumber}

The difference comes from the fact that \lstinline|signed long int| is 32-bit
wide for \lstinline|x86_32|, and 64-bit wide for \lstinline|x86_64|.

\end{example}

\subsection{Signed downcasting}

Note that arithmetic operations usually involve arithmetic conversions.  For
instance, integer expressions with rank lower than \lstinline|int| are promoted,
thus the following program:

\smallskip

\begin{listing-nonumber}
int main(void) {
  signed char cx, cy, cz;

  cz = cx + cy;
  return 0;
}
\end{listing-nonumber}

\smallskip
is in fact equivalent to:

\smallskip
\begin{listing-nonumber}
int main(void) {
  signed char cx, cy, cz;

  cz = (signed char)((int)cx + (int)cy);
  return 0;
}
\end{listing-nonumber}

Since a signed overflow can occur on expression \lstinline|(int)cx + (int)cy|,
the following annotations are generated by the \rte{} plug-in:
\begin{listing-nonumber}
/*@ assert rte: signed_overflow: -2147483648 <= (int)cx+(int)cy; */
/*@ assert rte: signed_overflow: (int)cx+(int)cy <= 2147483647; */
\end{listing-nonumber}

This is much less constraining than what one would want to infer, namely:
\begin{listing-nonumber}
/*@ assert (int)cx+(int)cy <= 127; */
/*@ assert -128 <= (int)cx+(int)cy; */
\end{listing-nonumber}

Actually, by setting the option \texttt{-warn-signed-downcast} (which is unset
by default), the \rte{} plug-in infers these second (stronger) assertions when
treating the cast of the expression to a \lstinline|signed char|.  Since the
value represented by the expression cannot in general be represented as a
\lstinline|signed char|, and following \cnn{} paragraph \mbox{6.3.1.3.3} (on
downcasting to a signed type), an {\it implementation-defined behavior} happens
whenever the result falls outside the range \lstinline|[-128,127]|.  Thus, with
a single annotation, the \rte{} plug-in prevents both an undefined behavior
(signed overflow) and an implementation defined behavior (signed downcasting).
Note that the annotation for signed downcasting always entails the annotation
for signed overflow.

\subsection{Unary minus}

The only case when a (signed) unary minus integer expression \lstinline|-expr|
overflows is when \lstinline|expr| is equal to the minimum value of the integer
type. Thus the generated assertion is as follows:
\begin{listing-nonumber}
int ix;
// some code
/*@ assert rte: signed_overflow: -2147483647 <= ix; */
ix = - ix;
\end{listing-nonumber}

\subsection{Division and modulo}

As of \cnn{} paragraph \mbox{6.5.5}, an undefined behavior occurs whenever the
value of the second operand of operators \lstinline|/| and \lstinline|%| is
zero. The corresponding runtime error is usually referred to as ``division by
zero''.  This may happen for both signed and unsigned operations.

\begin{listing-nonumber}
unsigned int ux;
// some code
/*@ assert rte: division_by_zero: ux != 0; */
ux = 1 / ux;
\end{listing-nonumber}

In 2's complement representation and for signed division, dividing the minimum
value of an integer type by $-1$ overflows , since it would give the maximum
value plus one.  There is no such rule for signed modulo, since the result would
be zero, which does not overflow.

\begin{listing-nonumber}
int x,y,z;
// some code
/*@ assert rte: division_by_zero: x != 0; */
/*@ assert rte: signed_overflow: y/x <= 2147483647; */
z = y / x;
\end{listing-nonumber}


\subsection{Bitwise shift operators}

\cnn{} paragraph \mbox{6.5.7} defines undefined and implementation defined
behaviors for bitwise shift operators.  The type of the result is the type of
the promoted left operand.

The undefined behaviors are the following:
\begin{itemize}
\item the value of the right operand is negative or is greater than or equal to
  the width of the promoted left operand:

\begin{listing-nonumber}
int x,y,z;

/*@ assert rte: shift: 0 <= y < 32; */
z = x << y; // same annotation for z = x >> y;
\end{listing-nonumber}

\item in \lstinline|E1 << E2|, \lstinline|E1| has signed type and negative
  value:

\begin{listing-nonumber}
int x,y,z;

/*@ assert rte: shift: 0 <= x; */
z = x << y;
\end{listing-nonumber}

\item in \lstinline|E1 << E2|, \lstinline|E1| has signed type and nonnegative
  value, but the value of the result $\lstinline|E1| \times 2^{\lstinline|E2|}$
  is not representable in the result type:

\begin{listing-nonumber}
int x,y,z;

/*@ assert rte: signed_overflow: x<<y <= 2147483647; */
z = x << y;
\end{listing-nonumber}

\end{itemize}

There is also an implementation defined behavior if in \lstinline|E1 >> E2|,
\lstinline|E1| has signed type and negative value.  This case corresponds to the
arithmetic right-shift, usually defined as signed division by a power of two,
with two possible implementations: either by rounding the result towards minus
infinity (which is standard) or by rounding towards zero. \rte{} generates an
annotation for this implementation defined behavior.

\begin{listing-nonumber}
int x,y,z;

/*@ assert rte: shift: 0 <= x; */
z = x << y;
\end{listing-nonumber}

\begin{example} ~
The following example summarizes \rte{} generated annotations for bitwise shift
operations, with \lstinline|-machdep x86_64|:

\begin{listing-nonumber}
long x,y,z;

/*@ assert rte: shift: 0 <= y < 64; */
/*@ assert rte: shift: 0 <= x; */
/*@ assert rte: signed_overflow: x<<y <= 9223372036854775807; */
z = x << y;

/*@ assert rte: shift: 0 <= y < 64; */
/*@ assert rte: shift: 0 <= x; */
z = x >> y;
\end{listing-nonumber}

\end{example}

\section{Left-values access}

Dereferencing a pointer is an undefined behavior if:
\begin{itemize}

\item the pointer has an invalid value: null pointer, misaligned address for the
  type of object pointed to, address of an object after the end of its lifetime
  (see \cnn{} paragraph \mbox{6.5.3.2.4});

\item the pointer points one past the last element of an array object: such a
  pointer has a valid value, but should not be dereferenced (\cnn{} paragraph
  \mbox{6.5.6.8}).
\end{itemize}

The \rte{} plug-in generates annotations to prevent this type of undefined
behavior in a systematic way. It does so by deferring the check to the \acsl{}
built-in predicate \lstinline|valid(p)|: \lstinline|valid(s)| (where
\lstinline|s| is a set of terms) holds if and only if dereferencing any
$\lstinline|p| \in \lstinline|s|$ is safe (i.e. points to a safely allocated
memory location).  A distinction is made for read accesses, that generate
\lstinline|\valid_read(p)| assertions (the locations must be at least readable),
and write accesses, for which \lstinline|\valid(p)| annotations are emitted (the
locations must be readable and writable).

Since an array subscripting \lstinline|E1[E2]| is identical to
\lstinline|(*((E1) + (E2)))| (\cnn{} paragraph \mbox{6.5.2.1.2}), the ``invalid
access'' undefined behaviors naturally extend to array indexing, and \rte{} will
generate similar annotations.  However, when the array is known, \rte{} attempts
to generate simpler assertions.  Typically, on an access \lstinline|t[i]| where
\lstinline|t| has size \lstinline|10|, \rte{} will generate two assertions
\lstinline|0 <= i| and \lstinline|i < 10|, instead of \lstinline|\valid(&t[i])|.

The kernel option \lstinline|-safe-arrays| (or \lstinline|-unsafe-arrays|)
influences the annotations that are generated for an access to a
multi-dimensional array, or to an array embedded in a struct.  Option
\lstinline|-safe-arrays|, which is set by default in Frama-C, requires that all
syntactic accesses to such an array remain in bound. Thus, if the field
\lstinline|t| of the struct \lstinline|s| has size \lstinline|10|, the access
\lstinline|s.t[i]| will generate an annotation \lstinline|i < 10|, even if some
fields exist after \lstinline|t| in \lstinline|s|.\footnote{ Thus, by default,
  RTE is more stringent than the norm. Use option \lstinline|-unsafe-arrays| if
  you want to allow code such as \lstinline|s.t[12]| in the example above.}
Similarly, if \lstinline|t| is declared as \lstinline|int t[10][10]|, the access
\lstinline|t[i][j]| will generate assertions \lstinline|0 <= i < 10| and
\lstinline|0 <= j < 10|, even though \lstinline|t[0][12]| is also
\lstinline|t[1][2]|.

Finally, dereferencing a pointer to a functions leads to the emission of
a \lstinline|\valid_function| predicate, to protect against a possibly
invalid pointer (\cnn{} 6.3.2.3:8). Those assertions are generated provided
option \lstinline|-rte-pointer-call| is set.



\begin{example} ~
An example of \rte{} annotation generation for checking the validity of each
memory access:
\begin{listing-nonumber}
extern void f(int* p);
int i;
unsigned int j;

int main(void) {
  int *p;
  int tab[10];

  /*@ assert rte: mem_access: \valid(p); */
  *p = 3;

  /*@ assert rte: index_bound: 0 <= i; */
  /*@ assert rte: index_bound: i < 10; */
  /*@ assert rte: mem_access: \valid_read(p); */
  tab[i] = *p;

  /*@ assert rte: mem_access: \valid(p+1); */
  /*@ assert rte: index_bound: j < 10; */
  // No annotation 0 <= j, as j is unsigned
  *(p + 1) = tab[j];

  return 0;
}
\end{listing-nonumber}

% Note that in the call \lstinline|f(tab)|, the implicit conversion from array \lstinline|tab| to a pointer to the beginning of the array
% \lstinline|&tab[0]| introduces a pointer dereferencing and thus the annotation \lstinline|\valid((int*) tab)|, which is equivalent to
% \lstinline|\valid(&tab[0])|.

\end{example}

\begin{example} ~
An example of memory access validity annotation generation for structured types,
with options \lstinline|-safe-arrays| and \lstinline|-rte-pointer-call| set.

\begin{listing-nonumber}
struct S {
   int val;
   struct S *next;
};

struct C {
   struct S cell[5];
   int (*f)(int);
};

struct ArrayStruct {
   struct C data[10];
};

unsigned int i, j;

int main() {
  int a;
  struct ArrayStruct buff;
  // some code

  /*@ assert rte: index_bound: i < 10; */
  /*@ assert rte: index_bound: j < 5; */
  /*@ assert rte: mem_access: \valid_read(&(buff.data[i].cell[j].next)->val); */
  a = (buff.data[i].cell[j].next)->val;

  /*@ assert rte: index_bound: i < 10; */
  /*@ assert rte: function_pointer: \valid_function(buff.data[i].f); */
  (*(buff.data[i].f))(a);

  return 0;
}
\end{listing-nonumber}

Notice the annotation generated for the call \lstinline|(*(buff.data[i].f))(a)|.

\end{example}

%%\section{String literal modification}

%%6.4.5 (not so frequent)

\section{Unsigned overflow annotations}

\cnn{} states that {\it unsigned} integer arithmetic is modular: overflows do
not occur (paragraph \mbox{6.2.5.9} of \cnn{}).  On the other hand, most
first-order solvers used in deductive verification (excluding dedicated
bit-vector solvers such as \cite{Boolector}) either provide only non-modular
arithmetic operators, or are much more efficient when no modulo operation is
used besides classic full-precision arithmetic operators. Therefore \rte{}
offers a way to generate assertions preventing unsigned arithmetic operations to
overflow ({\it i.e.} involving computation of a modulo).

Operations which are considered by \rte{} regarding unsigned overflows are
addition, subtraction, multiplication. Negation (unary minus), left shift.
and right shift are not considered. The generated assertion requires the result
of the operation (in non-modular arithmetic) to be less than the maximal
representable value of its type, and nonnegative (for subtraction).

\begin{example} ~

The following file only contains unsigned arithmetic operations: no assertion is
generated by \rte{} by default.
\begin{listing-nonumber}
unsigned int f(unsigned int a, unsigned int b) {
  unsigned int x, y;
  x = a * (unsigned int)2;
  y = b - x;
  return y;
}
\end{listing-nonumber}

To generate assertions w.r.t. unsigned overflows, options
\lstinline|-warn-unsigned-overflow| must be used. Here is the resulting
file on a 32 bits target architecture (\lstinline|-machdep x86_32|):
\begin{listing-nonumber}
unsigned int f(unsigned int a, unsigned int b) {
  unsigned int x, y;
  /*@ assert rte: unsigned_overflow: 0 <= a*(unsigned int)2; */
  /*@ assert rte: unsigned_overflow: a*(unsigned int)2 <= 4294967295; */
  x = a * (unsigned int)2;
  /*@ assert rte: unsigned_overflow: 0 <= b-x; */
  /*@ assert rte: unsigned_overflow: b-x <= 4294967295; */
  y = b - x;
  return y;
}


\end{listing-nonumber}
\end{example}

\section{Unsigned downcast annotations}

Downcasting an integer type to an unsigned type is a well-defined behavior,
since the value is converted using a modulo operation just as for unsigned
overflows (\cnn{} paragraph {6.3.1.3.2}). The \rte{} plug-in offers the
possibility to generate assertions preventing such occurrences of modular
operations with the \lstinline|-warn-unsigned-downcast| option.

\begin{example} ~

On the following example, the sum of two \lstinline|int| is returned as an
unsigned char:

\begin{listing-nonumber}
unsigned char f(int a, int b) {
  return a+b;
}
\end{listing-nonumber}

Using \rte{} with the \lstinline|-warn-unsigned-downcast| option gives the
following result:
\begin{listing-nonumber}
unsigned char f(int a, int b) {
  unsigned char __retres;
  /*@ assert rte: unsigned_downcast: a+b <= 255; */
  /*@ assert rte: unsigned_downcast: 0 <= a+b; */
  /*@ assert rte: signed_overflow: -2147483648 <= a+b; */
  /*@ assert rte: signed_overflow: a+b <= 2147483647; */
  __retres = (unsigned char)(a + b);
  return (__retres);
}
\end{listing-nonumber}


\end{example}

\section{Cast from floating-point to integer types}

Casting a value from a real floating type to an integer type is
allowed only if the value fits within the integer range (ISO C99
paragraph \mbox{6.3.1.4}), the conversion being done with a truncation
towards zero semantics for the fractional part of the real floating
value.  The \rte{} plug-in generates annotations that ensure that no
undefined behavior can occur on such casts.

\begin{listing-nonumber}
int f(float v) {
  int i = (int)(v+3.0f);
  return i;
}
\end{listing-nonumber}

Using \rte{} with the \lstinline|-rte-float-to-int| option, which is set
by default, gives the following result:
\begin{listing-nonumber}
int f(float v) {
  int i;
  /*@ assert rte: float_to_int: v+3.0f < 2147483648; */
  /*@ assert rte: float_to_int: -2147483649 < v+3.0f; */
  i = (int)(v + 3.0f);
  return i;
}
\end{listing-nonumber}


\section{Expressions not considered by \rte{}}

An expression which is the operand of a \lstinline|sizeof| (or
\lstinline|__alignof|, a \gcc{} operator parsed by Cil) is ignored by \rte{}, as
are all its sub-expressions.  This is an approximation, since the operand of
\lstinline|sizeof| may sometimes be evaluated at runtime, for instance on
variable sized arrays: see the example in \cnn{} paragraph \mbox{6.5.3.4.7}.
Still, the transformation performed by Cil on the source code actually ends up
with a statically evaluated \lstinline|sizeof| (see the example below).  Thus
the approximation performed by \rte{} seems to be on the safe side.

\begin{example} ~
Initial source code:

\begin{listing-nonumber}
#include <stddef.h>

size_t fsize3(int n) {
  char b[n + 3]; // variable length array
  return sizeof b; // execution time sizeof
}

int main() {
  return fsize3(5);
}
\end{listing-nonumber}

Output obtained with \lstinline|frama-c -print| with \lstinline|gcc|
preprocessing:

\begin{listing-nonumber}
typedef unsigned long size_t;
/* compiler builtin:
   void *__builtin_alloca(unsigned int);   */
size_t fsize3(int n)
{
  size_t __retres;
  char *b;
  unsigned int __lengthofb;
  {
    /*undefined sequence*/
    __lengthofb = (unsigned int)(n + 3);
    b = (char *)__builtin_alloca(sizeof(*b) * __lengthofb);
  }
  __retres = (unsigned long)(sizeof(*b) * __lengthofb);
  return __retres;
}

int main(void)
{
  int __retres;
  size_t tmp;
  tmp = fsize3(5);
  __retres = (int)tmp;
  return __retres;
}
\end{listing-nonumber}

\end{example}

\section{Initialization}

Reading an uninitialized value can be an undefined behavior in the two following
cases:

\begin{itemize}
\item \mbox{ISO C11 6.3.2.1}\footnote{here we use C11 as it is clearer than C99}
      a variable whose address is never taken is read before being initialized,
\item a memory location that has never been initialized is read and it happens
      that it was a trap representation for the type used for the access.
\end{itemize}

More generally, reading an uninitialized location always results in an
indeterminate value (\mbox{6.7.9.10}). Such a value is either an unspecified
value or a trap representation. Only reading a trap representation is an
undefined behavior (\mbox{6.2.6.1.5}). It corresponds to the second case above.

However for (most) types that do not have trap representation, reading an
unspecified value is generally not a desirable behavior. Thus, \rte{} is
stricter than the ISO C on many aspects and delegates one case of undefined
behavior to the use of compiler warnings. We now detail the chosen tradeoff.

If a value of a fundamental type (integers, floating point types, pointers, or
a typedef of those) is read, it must be initialized except if it is a formal
parameter or a global variable. We exclude formal parameters as its
initialization status must be checked at the call point (\rte{} generates an
annotation for this). We exclude global variables as they are initialized by
default and any value stored in this variable must be initialized (\rte{}
generates an annotation for this).

As structures and unions never have trap representation, they can (and they are
regularly) be manipulated while being partially initialized. Consequently,
\rte{} does not require initialization for reads of a full union or structure
(while reading fields with fundamental types is covered by the previous paragraph).
As a consequence, the case of structures and unions \textit{whose address is
 never taken, and being read before being initialized} is \textbf{not} covered
by \rte{}. It is worth noting that
this particular case is efficiently detected by a compiler warning (see
\lstinline{-Wuninitialized} on \gcc{} and \clang{} for example) as it only
requires local reasoning that is easy for compilers (but not that simple to
express in \acsl{}).

If you really need \rte{} to cover the previous case, please contact the
Frama-C team.

Finally, there are some cases when reading uninitialized values via a type that
cannot have trap representation (for example \lstinline{unsigned char}) should
be allowed, for example writing a \lstinline{memcpy} function. In this case, one
can exclude this function from the range of function annotated with
initialization properties by removing them from the set of functions to annotate
(e.g. \lstinline{-rte-initialized="@all,-f"}). Note that the excluded functions
must preserve the initialization of globals, as no assertions are generated for
them.

\section{Undefined behaviors not covered by \rte{}}

One should be aware that \rte{} only covers a small subset of all possible
undefined behaviors (see annex J.2 of \cite{standardc99} for a complete list).

In particular, undefined behaviors related to the following operations are not
considered:

\begin{itemize}
\item Use of relational operators for the comparison of pointers that do not
  point to the same aggregate or union (\cnn{} 6.5.8)
\item Demotion of a real floating type to a smaller floating type
  producing a value outside of the representable range (\cnn{} 6.3.1.5)
\item Conversion between two pointer types produces a result that is incorrectly
  aligned (\cnn{} 6.3.2.3)
\end{itemize}

%% \Section{Others}
%% ISO 6.3.1.3 / 6.3.1.4 / 6.3.1.5
%% convert an integer type to another signed integer type that cannot represent its value: implementation defined.
%% convert a real floating type to an integer: if the value of the integral part cannot be represented by the integer type, undefined.
%% convert an integer to a real floating type :
%% if the value being converted is outside the range of values that can be represented,
%% undefined (does not happen with IEEE floating types, event if real floating = float and integer type = unsigned long long).
%% If in range but not exact, round to nearest higher or nearest lower representable value (implementation defined).
%% Value analysis rounds to nearest lower silently (?).
%% demote a real floating type to another and procuce a value outside the range = undefined

\chapter{Plug-in Options}

Enabling \rte{} plug-in is done by adding \lstinline|-rte| on the command-line
of Frama-C. The plug-in then selects every C function which is in the set
defined by the \lstinline|-rte-select|: if no explicit set of functions is
provided by the user, all C functions defined in the program are selected.
Selecting the kind of annotations which will be generated is performed by using
other \rte{} options:
\begin{description}
\item[{\tt -rte} (boolean, defaults to false)] \ \smallskip \\
     Enable \rte{} plug-in
\item[{\tt -rte-div} (boolean, defaults to true)] \ \smallskip \\
     Generate annotations for division by zero
\item[{\tt -rte-float-to-int} (boolean (defaults to true))] \ \smallskip \\
     Generate annotations for casts from floating-point to integer
\item[{\tt -rte-initialized} (set of function (defaults to none))] \ \smallskip \\
     Generate annotations for initialization for the given set of functions
\item[{\tt -rte-mem} (boolean (defaults to true))] \ \smallskip \\
     Generate annotations for validity of left-values access
\item[{\tt -rte-pointer-call} (boolean (defaults to true))] \ \smallskip \\
     Generate annotations for validity of calls via function pointers
\item[{\tt -rte-shift} (boolean (defaults to true))] \ \smallskip \\
     Generate annotations for left and right shift value out of bounds
\item[{\tt -rte-select} (set of function (defaults to all))] \ \smallskip \\
     Run plug-in on a subset of C functions
\item[{\tt -rte-trivial-annotations} (boolean (defaults to false))] \ \smallskip \\
     Generate all annotations even when they trivially hold
\item[{\tt -rte-warn} (boolean (defaults to true))] \ \smallskip \\
     Emit warning on broken annotations
\end{description}
Combined with the following Kernel options:
\begin{description}
\item[{\tt -warn-unsigned-overflow} (boolean, defaults to false)] \ \smallskip \\
     Generate annotations for unsigned overflows
\item[{\tt -warn-unsigned-downcast} (boolean, defaults to false)] \ \smallskip \\
     Generate annotations for unsigned integer downcast
\item[{\tt -warn-signed-overflow} (boolean, defaults to true)] \ \smallskip \\
     Generate annotations for signed overflows
\item[{\tt -warn-signed-downcast} (boolean, defaults to false)] \ \smallskip \\
     Generate annotations for signed integer downcast
\item[{\tt -warn-pointer-downcast} (boolean, defaults to true)] \ \smallskip \\
     Generate annotations for downcast of pointer values
\item[{\tt -warn-left-shift-negative} (boolean, defaults to true)] \ \smallskip \\
     Generate annotations for left shift on negative values
\item[{\tt -warn-right-shift-negative} (boolean, defaults to false)] \ \smallskip \\
     Generate annotations for right shift on negative values
\item[{\tt -warn-invalid-bool} (boolean, defaults to true)] \ \smallskip \\
     Generate annotations for \lstinline|_Bool| trap representations
\item[{\tt -warn-special-float} (string: {\tt non-finite} (default), {\tt nan} or {\tt none})] \ \smallskip \\
     Generate annotations when special floats are produced: infinite floats or
     NaN (by default), only on NaN or never.
\item[{\tt -warn-invalid-pointer} (boolean, defaults to false)] \ \smallskip \\
     Generate annotations for invalid pointer
     arithmetic\footnote{For the rationale about this default value, please refer to
     the \framac{} user manual~\cite{userman}.}
\end{description}

Pretty-printing the output of \rte{} and relaunching the plug-in on the
resulting file will generate duplicated annotations, since the plug-in does not
check existing annotations before generation. This behaviour does not happen if
\rte{} is used in the context of a Frama-C project~\cite{framacdev}: the
annotations are not generated twice.




\cleardoublepage
\phantomsection
\addcontentsline{toc}{chapter}{\bibname}
\bibliographystyle{plain}
\bibliography{./biblio}
