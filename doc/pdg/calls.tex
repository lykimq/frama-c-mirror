\chapter{Dépendances interprocédurales}\label{sec-intro-call}

On a vu qu'un PDG est associé à une fonction.
La question se pose donc de savoir
comment calculer des dépendances interprocédurales, c'est-à-dire comment mettre
en relation les appels de fonctions et les dépendances des fonctions appelées.

Nous allons tout d'abord voir qu'un appel de fonction
est représenté par plusieurs éléments dans le PDG (\S\ref{sec-call}).
Puis, nous allons voir que pour mettre en relation des appels et les fonctions
appelées, ils faut ajouter d'autres éléments à chaque fonction
(\S\ref{sec-fct-inout}).

\section{Appels de fonction}\label{sec-call}

L'instruction contenant l'appel est représentée par plusieurs
éléments dans le graphe de dépendances
afin de pouvoir plus précisément mettre en relation les appels aux
fonctions appelées.

Les éléments créés sont les suivants~:
\begin{itemize}
\item un élément pour chaque paramètre de la fonction appelée;
  les dépendances sont crées par une simulation
  de l'affectation des arguments d'appel dans les paramètres formels,
\item un élément représentant le contrôle du point d'entrée de la fonction
  appelée (un peu comme si l'appel était dans un bloc et que ce noeud
  représentait ce bloc),
\item un élément pour chaque sortie, dépendant des entrées correspondantes.
\end{itemize}

Pour ne pas avoir à calculer les flots de données de toutes les fonctions de
l'application, il a été décidé d'utiliser les dépendances ({\it from})
calculées indépendamment par \ppc.
La liste des entrées et des sorties, ainsi que les dépendances entre les unes et
les autres sont extraites des spécifications des fonctions appelées, et non de
leur PDG\footnote{c'est peut-être un problème
si on fait de la coupure de branche, car les dépendances peuvent être réduites
par une telle spécialisation.}.
Ceci est vrai également pour les
fonctions dont le code est absent de l'application étudiée
car cela permet d'être cohérent avec les autres analyses.
Cela permettra en particulier, d'utiliser d'éventuelles propriétés
fournies par la suite par l'utilisateur.

\begin{exemple}
\begin{tabular}{m{0.35\textwidth} m{0.59\textwidth}}
\begin{verbatim}
struct {int a;
        int b; } G;

/*@ assigns \result {a},
            G.a {G, a} */
int g (int a);

int f (int x, int y) {
  G.b = x;
  x = g (x+y);
  return x + G.b;
}
\end{verbatim}
&
Ici, pour représenter l'appel à \verbtt{g} dans \verbtt{f} dans le PDG,
on va avoir~:
\begin{itemize}
  \item un élément représentant le point d'entrée dans \verbtt{g},
  \item un élément $e_1$ pour représenter \verbtt{a = x+y},
    c'est-à-dire l'affectation de l'argument de l'appel
    dans le paramètre formel de \verbtt{g},
  \item un élément $e_2$ pour calculer la valeur de retour de \verbtt{g},
    qui dépend de la valeur de $e_1$
    (utilisation de la spécification de $g$),

\item et enfin, un élément $e_3$ qui représente la seconde sortie de \verbtt{g}~:
\verbtt{G.a} qui dépend du paramètre \verbtt{a} et donc de $e_1$
    et de des éléments $\{ e_G \}$
    correspondant à
    la valeur de $G$ avant l'appel
(selon la spécification de $g$).
\end{itemize}
\end{tabular}
\end{exemple}

On note que, contrairement à ce qui était fait dans la version précédente,
on ne crée par d'élément pour l'entrée implicite $G$ de $g$ dans $f$.
Cela permet d'améliorer la précision des dépendances lorsque
l'ajout d'un tel noeud conduisait au regroupement de plusieurs données.

Ainsi, dans l'exemple précédent, on ne crée pas d'élément pour
représenter la valeur de $G$ avant l'appel, même si l'élément $e_3$ en dépend,
et on conserve donc l'information que $G.b$ ne dépend que de l'affectation
précédent l'appel.


\section{Entrées/sorties d'une fonction}\label{sec-fct-inout}

Pour relier un appel de fonction au PDG de la fonction appelée,
il faut ajouter des éléments représentant ses entrées/sorties,
c'est-à-dire~:

\begin{itemize}
  \item un élément correspondant au point de contrôle d'entrée
dans la fonction,
\item deux éléments pour chaque paramètre (cf. \S\ref{sec-decl-param}),
\item un élément pour les entrées implicites (cf. \S\ref{sec-impl-in}),
\item un élément pour la sortie de la fonction si celle-ci retourne
  quelque chose.
\end{itemize}

On note que, contrairement à ce qui était fait dans la version précédente,
on ne crée par d'élément pour les sorties implicites de la fonction.
Cela permet d'améliorer la précision des dépendances lorsque
l'ajout d'un tel noeud conduisait au regroupement de plusieurs données.

C'est par exemple le cas lorsqu'une fonction calcule $G.a$,
puis $G.a.x$ car un élément de sortie regrouperait les deux alors que
si par la suite on s'intéresse juste à $G.a.x$ à la sortie de la fonction,
le fait de ne pas avoir créé cet élément permet de retrouver l'information plus
précise.

\subsection{Entrées implicites}\label{sec-impl-in}

Au cours du calcul du PDG, on mémorise l'utilisation des données
qui ne sont pas préalablement définies.
Cela permet par la suite que créer des éléments pour ces entrées dites
implicites. On ne crée pas d'élément pour les variables locales non
initialisées, mais un message d'avertissement est émis.
Il est possible que ce soit une fausse alerte dans le cas où l'initialisation
est faite dans une branche dont la condition est forcement vrai à chaque
exécution où l'on passe par la suite par l'utilisation.

Diverses stratégies de regroupement de ces entrées peuvent être utilisées.
A ce jour, l'outil construit tous les éléments lui permettant d'avoir une
meilleure précision. C'est-à-dire que deux éléments peuvent représenter les
données qui ont une intersection.

\subsection{Déclaration des paramètres formels}\label{sec-decl-param}

En plus de l'élément représentant la valeur des paramètres,
on crée un second élément qui représente sa déclaration,
le premier dépendant du second.

Cette représentation peut permettre d'avoir une meilleure précision
dans le cas où certains calcul ne dépendent pas de la valeur du
paramètre, mais uniquement de sa déclaration.

\begin{exemple}
\begin{tabular}{m{5cm} m{\dimexpr\linewidth-6cm}}
\begin{verbatim}
int g (int a) {
  G = 2 * a;
  a = calcul_a ();
  return a;
}
int f (void) {
  int x = calcul_x ();
  return g (x);
}
\end{verbatim}
&
On voit que la valeur de retour de \verbtt{g} ne nécessite pas la valeur initiale
de \verbtt{a}, mais seulement sa déclaration. La valeur de retour de \verbtt{f}
ne dépend donc pas de l'appel à \verbtt{calcul\_x}.
\end{tabular}
\end{exemple}

Ce point n'est pas encore implémenté dans la version actuelle,
car dans des cas plus complexe, il est délicat de savoir ce qu'il faut
garder dans la fonction appelante. Le plus simple serait sans doute
de transformer le paramètre formel en une variable locale,
mais le filtrage permet à l'heure actuelle de garder ou d'effacer des
éléments existants, mais pas d'effectuer des transformations de code...

\section{Fonctions à nombre d'arguments variable}

Pour l'instant, on ne calcule pas le PDG des fonctions à nombre
d'arguments variable, c'est-à-dire que pour le reste de l'application,
tout se passe comme si on n'avait pas le code source de ces fonctions.\\

En revanche, les appels à de telles fonctions sont gérées de manière semblable à
ce qui est fait pour les autres appels, c'est-à-dire~:
\begin{itemize}
  \item création d'un noeud d'entrée pour chaque argument d'appel,
    (il y en a donc éventuellement plus que que paramètres formels dans le
    déclaration de la fonction appelée)
  \item utilisation des informations {\it from} pour créer les éventuelles
    entrées implicites, les sorties, et les liens de dépendance.
\end{itemize}

\section{Exemple}

\begin{exemple}

\lstinputlisting[language=c]{exple-call.c}
\end{exemple}

Graphe de la fonction \verbtt{f}: \\

\includegraphics[width=0.6\textwidth]{call-f}
\\

\clearpage

Graphe de la fonction \verbtt{g}: \\

\includegraphics[width=.9\textwidth]{call-g}
\\

Les graphes sont ceux qui sont effectivement produits par l'outil.
