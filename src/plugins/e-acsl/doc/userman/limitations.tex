\chapter{Known Limitations}

The development of the \eacsl plug-in is still ongoing. First, the \eacsl
reference manual~\cite{eacsl} is not yet fully supported. Which annotations can
already be translated into \C code and which cannot is defined in a separate
document~\cite{eacsl-implem}. Second, even though we do our best to avoid them,
bugs may exist. If you find a new one, please report it on the bug tracking
system\footnote{\url{https://git.frama-c.com/pub/frama-c/-/issues}} (see Chapter
10 of the \framac User Manual~\cite{userman}). Third, there are some additional
known limitations, which could be annoying for the user in some cases, but are
tedious to lift. Please contact us if you are interested in lifting these
limitations\footnote{Read
  \url{https://git.frama-c.com/pub/frama-c/blob/master/CONTRIBUTING.md} for
  additional details.}.

\section{Supported Systems}

\begin{important}
The only well-supported system is a Linux distribution on a 64-bit architecture.
\end{important}

\subsection{Operating Systems}

Non-Linux systems are almost not yet experimented. It might work but there are
most probably issues, in particular if using a non-standard
libc\index{Libc}. For instance, there are known bugs under Mac OS
X\index{Mac OS X}\footnote{See for instance at
  \url{https://bts.frama-c.com/view.php?id=2369}}.

\subsection{Architectures and Runtime Library}

The segment-based memory model (used by default for monitoring memory properties
such as \lstinline|\valid|) assumes little-endian architecture and has very
limited support for 32-bit architectures. When using a 32-bit machine or
big-endians, we recomend using the bittree-based memory model instead.

\begin{important}
The runtime library is also \emph{not} thread-safe.
\end{important}

\section{Uninitialized Values}
\index{Uninitialized value}

As explained in Section~\ref{sec:runtime-error}, the \eacsl plug-in should never
translate an annotation into \C code which can lead to a runtime error. This is
enforced, except for uninitialized values which are values read before having
been written.

\listingname{uninitialized.i}
\cinput{examples/uninitialized.i}

If you generate the instrumented code, compile it, and finally execute it, you
may get no runtime error depending on your \C compiler, but the behavior is
actually undefined because the assertion reads the uninitialized variable
\lstinline|x|. You should be caught by the \eacsl plug-in, but that is not
the case yet.

\begin{logs}
\$ e-acsl-gcc.sh uninitialized.i -c -Omonitored_uninitialized
monitored_uninitialized.i: In function 'main':
monitored_uninitialized.i:44:16: warning: 'x' is used uninitialized in this function
[-Wuninitialized]
\$ ./monitored_uninitialized.e-acsl
\end{logs}

This is more a design choice than a limitation: should the \eacsl plug-in
generate additional instrumentation to prevent such values from being evaluated,
the generated code would be much more verbose and slower.

If you really want to track such uninitializations in your annotation, you have
to manually add calls to the \eacsl predicate
\lstinline|\initialized|~\cite{eacsl}.

\section{Incomplete Programs}

Section~\ref{sec:incomplete} explains how the \eacsl plug-in is able to handle
incomplete programs, which are either programs without main, or programs
containing undefined functions (\emph{i.e.} functions without body).

However, if such programs contain memory-related annotations, the generated code
may be incorrect. That is made explicit by a warning displayed when the \eacsl
plug-in is running (see examples of Sections~\ref{sec:no-main} and
\ref{sec:no-code}).

\subsection{Programs without Main}
\index{Program!Without main}
\label{sec:limits:no-main}

The instrumentation in the generated program is partial for every program
without main containing memory-related annotations, except if the option
\optionuse{-}{e-acsl-full-mtracking} or the \eacsl plug-in (of \shortopt{M}
option of \eacslgcc) is provided. In that case, violations of such annotations
are undetected.

Consider the following example.

\listingname{valid\_no\_main.c}
\cinput{examples/valid_no_main.c}

You can generate the instrumented program as follows.
\begin{logs}
\$ e-acsl-gcc.sh -M -omonitored_valid_no_main.i valid_no_main.c
[kernel] Parsing valid_no_main.c (with preprocessing)
[e-acsl] beginning translation.
[kernel] Parsing FRAMAC_SHARE/e-acsl/e_acsl.h (with preprocessing)
[kernel] Warning: no entry point specified:
  you must call functions `__e_acsl_globals_init', `__e_acsl_globals_clean',
  `__e_acsl_memory_init' and `__e_acsl_memory_clean' by yourself.
[e-acsl] translation done in project "e-acsl".
\end{logs}

The last warning states an important point: if this program is linked against
another file containing \texttt{main} function, then this main function must
be modified to insert a calls to the functions
\texttt{\_\_e\_acsl\_globals\_init}
\index{e\_acsl\_globals\_init@\texttt{\_\_e\_acsl\_globals\_init}} and
\texttt{\_\_e\_acsl\_memory\_init}
\index{e\_acsl\_memory\_init@\texttt{\_\_e\_acsl\_memory\_init}} at the very
beginning. These functions play a very important role: the latter initializes
metadata storage used for tracking of memory blocks while the former initializes
tracking of global variables and constants. Unless these calls are inserted the
run of a modified program is likely to fail.

While it is possible to add such intrumentation manually we recommend using
\eacslgcc. Consider the following incomplete program containing \T{main}:

\listingname{modified\_main.c}
\cinput{examples/modified_main.c}

Then just compile and run it as explained in Section~\ref{sec:memory}.

\begin{logs}
\$ e-acsl-gcc.sh -M -omonitored_modified_main.i modified_main.c
\$ e-acsl-gcc.sh -C -Ovalid_no_main monitored_modified_main.i monitored_valid_no_main.i
\$ ./valid_no_main.e-acsl
Assertion failed at line 11 in function f.
The failing predicate is:
freed: \valid(x).
Aborted
\end{logs}

Also, if the unprovided main initializes some variables, running the
instrumented code (linked against this main) could print some warnings from the
\eacsl memory library\footnote{see
  \url{https://bts.frama-c.com/view.php?id=1696} for an example}.

\subsection{Undefined Functions}
\label{sec:limits:no-code}
\index{Function!Undefined}

The instrumentation in the generated program is partial for a program $p$ if $p$
contains a memory-related annotation $a$ and an undefined function
$f$ such that:
\begin{itemize}
\item either $f$ has an (even indirect) effect on a left-value occurring in $a$;
\item or $a$ is one of the post-conditions of $f$.
\end{itemize}
A violation of such an annotation $a$ is undetected. There is no workaround yet.

\subsection{Incomplete Types}
\index{Type!Incomplete}

The instrumentation in the generated program is partial for a program $p$ if $p$
contains a memory-related annotation $a$ and a variable $v$ with an incomplete
type definition such that $a$ depends on $v$ (even indirectly).

A violation of such an annotation $a$ is undetected. There is no workaround yet.

\section{Recursive Functions}
\index{Function!Recursive}

Programs containing recursive functions have the same limitations as the ones
containing undefined functions (Section~\ref{sec:limits:no-code}) and
memory-related annotations.

%% JS: this issue should have been fixed:
%% Also, even though there is no such annotations, the generated code may call a
%% function before it is declared. When this behavior appears remains
%% unspecifed. The generated code is however easy to fix by hand.

\section{Variadic Functions}
\index{Function!Variadic}

Programs containing undefined variadic functions with contracts
are not yet supported. Using the \variadic plug-in of \framac could be a
solution in some cases, but its generated code cannot always be compiled.

\section{Function Pointers}
\index{Function!Pointer}

Programs containing function pointers have the same limitations on
memory-related annotations as the ones containing undefined or recursive
functions.

\section{Requirements to Input Programs}
\index{Function!Input}

\subsection{\eacsl Namespace}
While \eacsl uses source-to-source transformations and not binary
instrumentations it is important that the source code provided as input does
not contain any variables or functions prefixed \T{\_\_e\_acsl\_}.  \eacsl
reserves this namespace for its transformations, and therefore an input program
containing such symbols beforehand may fail to be instrumented or compiled.

\subsection{Memory Management Functions}
Programs providing custom definitions of \T{syscall}, \T{mmap} or
\T{sbrk} should be rejected. Also, an input programs should not modify
memory-management functions namely \T{malloc}, \T{calloc}, \T{realloc},
\T{free}, \T{cfree}, \T{posix\_memalign} and \T{aligned\_alloc}.  \eacsl relies
on these functions in order to track heap memory. Further, correct heap memory
monitoring requires to limit allocation and deallocation of heap memory to
POSIX-compliant memory management functions
listed above. Monitoring of programs that allocate memory using non-standard or
obsolete functions (e.g., \T{valloc}, \T{memalign}, \T{pvalloc}) may not work
correctly.
