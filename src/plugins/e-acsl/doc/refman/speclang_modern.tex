%; whizzy-master "main.tex"

\chapter{Specification language}
\label{chap:base}

%%%%%%%%%%%%%%%%%%%%%%%%%%%%%%%%%%%%%%%%%%%%%%%%%%%%%%%%%%%%%%%%%%%%%%%%%%%%%%%
%%%%%%%%%%%%%%%%%%%%%%%%%%%%%%%%%%%%%%%%%%%%%%%%%%%%%%%%%%%%%%%%%%%%%%%%%%%%%%%
%%%%%%%%%%%%%%%%%%%%%%%%%%%%%%%%%%%%%%%%%%%%%%%%%%%%%%%%%%%%%%%%%%%%%%%%%%%%%%%

\section{Lexical rules}
\nodiff

%%%%%%%%%%%%%%%%%%%%%%%%%%%%%%%%%%%%%%%%%%%%%%%%%%%%%%%%%%%%%%%%%%%%%%%%%%%%%%%
%%%%%%%%%%%%%%%%%%%%%%%%%%%%%%%%%%%%%%%%%%%%%%%%%%%%%%%%%%%%%%%%%%%%%%%%%%%%%%%
%%%%%%%%%%%%%%%%%%%%%%%%%%%%%%%%%%%%%%%%%%%%%%%%%%%%%%%%%%%%%%%%%%%%%%%%%%%%%%%

\section{Logic expressions}
\label{sec:expressions}

\except{the quantifications must be guarded}

More precisely, the grammars of terms and binders presented respectively
Figures~\ref{fig:gram:term} and~\ref{fig:gram:binders} are the same than the ones
of \acsl, while Figure~\ref{fig:gram:pred} presents the grammar of
predicates. The only differences introduced by \eacsl with respect to \acsl are
the fact that the quantifications that must be guarded and the introduction of
iterators.

\begin{figure}[htbp]
  \begin{cadre}
    \input{term_modern.bnf}
  \end{cadre}
  \caption{Grammar of terms. The terminals \emph{id}, \emph{C-type-name}, and
    various literals are the same as the corresponding \C lexical tokens.}
\label{fig:gram:term}
\end{figure}
\begin{figure}[htbp]
  \begin{cadre}
    \input{predicate_modern.bnf}
  \end{cadre}
  \caption{Grammar of predicates}
\label{fig:gram:pred}
\end{figure}
\begin{figure}[htbp]
  \fbox{\begin{minipage}{0.97\textwidth} \input{binders_modern.bnf}
    \end{minipage}}
  \caption{Grammar of binders and type expressions}
\label{fig:gram:binders}
\end{figure}

\subsection*{Quantification}

{\highlightnotimplemented The general form of quantifications (called
    generalized quantifications below), as described in
    Fig.~\ref{fig:gram:pred},
    is restricted to a few \emph{finite enumerable types}: the types of bound
    variables must be \C integer types, enum types, pointer types, or their
    aliases.

\unefficient{Generalized quantification over large types (for instance, types
  containing $2^{32}$ elements).}}

In addition to generalized quantifications, a restricted form of guarded
quantifications described in Fig.~\ref{fig:gram:guarded} is also recognized
\emph{for (possibly infinite) enumerable types} (typically,
\lstinline|integer|). In guarded quantifications, each bound variable must be
guarded exactly once and, if its bounds depend on other bound variables, these
variables must be guarded earlier or guarded by the same guard. Additionnally,
guards are limited to bound variables, meaning that the only allowed
identifiers $id$ are variable identifiers enclosed in the binder list.

\begin{figure}[htbp]
  \begin{cadre}
    \input{guarded_quantif_modern.bnf}
  \end{cadre}
  \caption{Grammar of guarded quantifications.}
  \label{fig:gram:guarded}
\end{figure}

\ifthenelse{\boolean{PrintImplementationRq}}{%
  \begin{notimplementedenv}
    Guarded quantifications over pointer types and enum types are not yet
    implemented.
  \end{notimplementedenv}
}

\begin{example}
  The following predicates are (labeled) guarded quantifications:
  \begin{itemize}
  \item \lstinline|sorted:| \Forall{} \lstinline|integer i, j; 0 <= i <= j < len ==> a[i] <= a[j]|
  \item {\highlightnotimplemented \lstinline|is_c:| \Exists{} \lstinline|u8 *q; p <= q < p + len && *q == (u8)c|}
  \end{itemize}
\end{example}

\subsection*{{\highlightnotimplemented{Iterator quantification}}} For
iterating over other data structures, \eacsl introduces a notion of $iterators$
over types that are introduced by a specific construct which attaches two sets
--- namely \texttt{nexts} and \texttt{guards} --- to a binary predicate over
a type $\tau$. This construct is
described by the grammar of Figure~\ref{fig:gram:iterator}.
  \begin{figure}[htbp]
    \begin{cadre}
      \input{iterator_modern.bnf}
    \end{cadre}
    \caption{Grammar of iterator declarations}
    \label{fig:gram:iterator}
  \end{figure}
  For a type $\tau$, \texttt{nexts} is a set of terms, and \texttt{guards} a set
  of predicates of the same cardinal. Each term in \texttt{nexts} is a function
  taking an argument of type $\tau$ and returning a value of type $\tau$ which
  is a successor of its argument. Each predicate in the set \texttt{guards}
  takes an
  element of type $\tau$, and is valid (resp. invalid) to indicate that the
  iteration should continue on the corresponding successor (resp.  stop at the
  given argument).

  Furthermore, the guard of a quantification using an iterator must be the
  predicate given in the definition of the iterator. This abstract binary
  predicate takes two arguments of the same type. One of them must be unnamed by
  using a wildcard (character underscore \texttt{'\_'}). The unnamed argument
  must be bound to the quantifier, while the other corresponds to the term from
  which the iteration begins.
  \begin{example}
    The following example introduces binary trees and a predicate which is valid
    if and only if each value of a binary tree is even.
    \cinput{link.c}
  \end{example}

%%%%%%%%%%%%%%%%%%%%%%%%%%%%%%%%%%%%%%%%%%%%%%%%%%%%%%%%%%%%%%%%%%%%%%%%%%%%%%%

\subsection{Operators precedence}
\nodiff

Figure~\ref{fig:precedence} summarizes operator precedences.
\begin{figure}[htbp]
  \begin{center}
    \begin{tabular}{|l|l|l|}
      \hline
      class 	& associativity & operators \\
      \hline
      selection & left & \lstinline|[$\cdots$]| \lstinline|->| \lstinline|.| \\
      unary 	& right & \lstinline|!| \lstinline|~| \lstinline|+|
      \lstinline|-| \lstinline|*| \lstinline|&| \lstinline|(cast)|
      \lstinline|sizeof| \\
      multiplicative & left & \lstinline|*| \lstinline|/|  \lstinline|%| \\
      additive & left & \lstinline|+| \lstinline|-| \\
      shift 	& left & \lstinline|<<| \lstinline|>>| \\
      comparison & left & \lstinline|<| \lstinline|<=| \lstinline|>| \lstinline|>=| \\
      comparison & left & \lstinline|==| \lstinline|!=| \\
      bitwise and & left & \lstinline|&| \\
      bitwise xor & left & \lstinline|^| \\
      bitwise or & left & \lstinline+|+ \\
      bitwise implies & left & \lstinline+-->+ \\
      bitwise equiv & left & \lstinline+<-->+ \\
      connective and     & left & \lstinline|&&| \\
      connective xor & left & \lstinline+^^+ \\
      connective or & left & \lstinline+||+ \\
      connective implies & right & \lstinline|==>| \\
      connective equiv & left & \lstinline|<==>| \\
      ternary connective & right & \lstinline|$\cdots$?$\cdots$:$\cdots$| \\
      binding & left & \Forall{} \Exists{} \Let{} \\
      naming & right & \lstinline|:| \\
      \hline
    \end{tabular}
  \end{center}
  \caption{Operator precedence}
\label{fig:precedence}
\end{figure}


%%%%%%%%%%%%%%%%%%%%%%%%%%%%%%%%%%%%%%%%%%%%%%%%%%%%%%%%%%%%%%%%%%%%%%%%%%%%%%%

\subsection{Semantics}
\label{sec:semantics}

\except{undefinedness and same laziness than \C}

More precisely, while \acsl is a 2-valued logic with only total
functions, \eacsl is a 3-valued logic with partial functions since
terms and predicates may be ``undefined''.

In this logic, the semantics of a term denoting a \C expression $e$ is undefined
if $e$ leads to a runtime error. Consequently the semantics of any term $t$
(resp. predicate $p$) containing a \C expression $e$ leading to a runtime error
is undefined if $e$ has to be evaluated in order to evaluate $t$ (resp. $p$).
\begin{remark}[Julien]
$e$ always terminates, thus no termination issue.
\end{remark}
\begin{example}
The semantics of all the below predicates are undefined:
\begin{itemize}
\item \lstinline|1/0 == 1/0|
\item \lstinline|f(*p)| for any logic function \lstinline|f| and invalid pointer
  \lstinline|p|
\end{itemize}
\end{example}

Furthermore, \C-like operators \lstinline|&&|, \lstinline+||+, and
\lstinline|_ ? _ : _| are lazy like in \C: their right members are evaluated 
only if required. Thus the amount of undefinedness is limited. Consequently,
predicate \lstinline|p ==> q| is also lazy since it is equivalent
to \lstinline+!p || q+. It is also the case for guarded quantifications since
guards are conjunctions and for ternary condition since it is equivalent to a
disjunction of implications.

\begin{example}\label{ex:semantics}
  All the predicates below are well defined. The first, second and fourth
  predicates are invalid, whereas the third one is valid:
\begin{itemize}
\item \lstinline|\false && 1/0 == 1/0|
\item \lstinline|\forall integer x, -1 <= x <= 1 ==> 1/x > 0|
\item \lstinline|\forall integer x, 0 <= x <= 0 ==> \false ==> -1 <= 1/x <= 1|
\item \lstinline|\exists integer x, 1 <= x <= 0 && -1 <= 1/0 <= 1|
\end{itemize}
In particular, the second one is invalid since the quantification is in fact an
enumeration over a finite number of elements, it amounts to
\lstinline|1/-1 > 0 && 1/0 > 0 && 1/1 > 0|. The first atomic proposition is
invalid, so the rest of the conjunction (and in particular 1/0) is not
evaluated. The fourth one is invalid since it is an existential quantification
over an empty range.

\emph{A contrario} the semantics of the predicates below is undefined:
\begin{itemize}
\item \lstinline|1/0 == 1/0 && \false|
\item \lstinline|-1 <= 1/0 <= 1 ==> \true|
\item \lstinline|\exists integer x, -1 <= x <= 1 && 1/x > 0 |
\end{itemize}
\end{example}

Furthermore, casting a term denoting a \C expression $e$ to a smaller type
$\tau$ is undefined if $e$ is not representable in $\tau$.

\begin{example}
Below, the first term is well-defined, while the second one is undefined.
\begin{itemize}
\item \lstinline|(char)127|
\item \lstinline|(char)128|
\end{itemize}
\end{example}

\paragraph{Handling undefinedness in tools}

It is the responsibility of each tool which interprets \eacsl to ensure that an
undefined term is never evaluated. For instance, it may exit with a proper
error message or, if it generates \C code, it may guard each generated
undefined \C expression in order to be sure that they are always safely used.

\ifthenelse{\boolean{PrintImplementationRq}}{%
  The \eacsl plug-in of \framac generates such guards. \notimplemented{Yet,
    a few guards are still missing.}

}

This behavior is consistent with both \acsl~\cite{acsl} and mainstream
specification languages for runtime assertion checking like
\jml~\cite{jml}. Consistency means that, if it exists and is defined, the \eacsl
predicate corresponding to a valid (resp. invalid) \acsl predicate is valid
(resp. invalid). Thus it is possible to reuse tools interpreting \acsl (e.g.,
\framac's \Eva~\cite{eva} or \Wp~\cite{wp} in order to interpret \eacsl, and
it is also possible to perform runtime assertion checking of \eacsl predicates
in the same way than \jml predicates. Reader interested by the implications
(especially issues) of such a choice may read the articles of Patrice
Chalin on that topic~\cite{chalin05,chalin07}.

%%%%%%%%%%%%%%%%%%%%%%%%%%%%%%%%%%%%%%%%%%%%%%%%%%%%%%%%%%%%%%%%%%%%%%%%%%%%%%%

\subsection{Typing}\label{sec:typing}
\nodiff

%%%%%%%%%%%%%%%%%%%%%%%%%%%%%%%%%%%%%%%%%%%%%%%%%%%%%%%%%%%%%%%%%%%%%%%%%%%%%%%

\subsection{Integer arithmetic and machine integers}
\nodiff

%%%%%%%%%%%%%%%%%%%%%%%%%%%%%%%%%%%%%%%%%%%%%%%%%%%%%%%%%%%%%%%%%%%%%%%%%%%%%%%

\subsection{Real numbers and floating point numbers}
\label{sec:reals}
\except{no quantification over real numbers and floating point numbers}

\difficults{Exact real numbers and even operations over floating point numbers}

More precisely, most real numbers are not representable at runtime with an
infinite precisions. Consequently, most operations over them (e.g., equality)
cannot be computed at runtime with an arbitrary precision. In such cases, it is
the responsibility of each tool which interprets \eacsl to specify the
level of precision (e.g., $1e^{-6}$) up to which it is sound, and/or to emit
undefinitive verdicts (e.g., ``unknown'').

\ifthenelse{\boolean{PrintImplementationRq}}{%
  \begin{notimplementedenv}
    Only floating point numbers (e.g., \texttt{0.1f}), rationals numbers (in
    $\mathbb{Q}$) and operations over them are supported by the \eacsl
    plug-in. Real numbers that are irrational numbers are not supported.
  \end{notimplementedenv}
}

%%%%%%%%%%%%%%%%%%%%%%%%%%%%%%%%%%%%%%%%%%%%%%%%%%%%%%%%%%%%%%%%%%%%%%%%%%%%%%%

\subsection{C arrays and pointers}
\nodiff

\difficultwhy{Ensuring validity of memory accesses}{the implementation of a
  memory model}

%%%%%%%%%%%%%%%%%%%%%%%%%%%%%%%%%%%%%%%%%%%%%%%%%%%%%%%%%%%%%%%%%%%%%%%%%%%%%%%

\subsection{Structures, Unions and Arrays in logic}
\label{sec:aggregates}
\nodiff

\difficults{Logic arrays without an explicit length}

\ifthenelse{\boolean{PrintImplementationRq}}{%
  \begin{notimplementedenv}
    The following constructs are currently not supported by the \eacsl plug-in:
    \begin{itemize}
    \item built-in function \Length;
    \item comparisons of unions and structures;
    \item functional updates;
    \item conversions of structure to structure.
    \end{itemize}
  \end{notimplementedenv}
}

%%%%%%%%%%%%%%%%%%%%%%%%%%%%%%%%%%%%%%%%%%%%%%%%%%%%%%%%%%%%%%%%%%%%%%%%%%%%%%%
%%%%%%%%%%%%%%%%%%%%%%%%%%%%%%%%%%%%%%%%%%%%%%%%%%%%%%%%%%%%%%%%%%%%%%%%%%%%%%%
%%%%%%%%%%%%%%%%%%%%%%%%%%%%%%%%%%%%%%%%%%%%%%%%%%%%%%%%%%%%%%%%%%%%%%%%%%%%%%%

\section{Function contracts}
\label{sec:fn-behavior}
\index{function contract}\index{contract}

\except{no clause \lstinline|terminates|}

Figure~\ref{fig:gram:contracts} shows the grammar of function contracts. This is
a simplified version of \acsl one without \lstinline|terminates|
clauses. Section~\ref{sec:termination} explains why \eacsl has no
\lstinline|terminates| clause.

\begin{figure}[htbp]
  \begin{cadre}
      \input{fn_behavior_modern.bnf}
   \end{cadre}
    \caption{Grammar of function contracts}
  \label{fig:gram:contracts}
\end{figure}

%%%%%%%%%%%%%%%%%%%%%%%%%%%%%%%%%%%%%%%%%%%%%%%%%%%%%%%%%%%%%%%%%%%%%%%%%%%%%%%

\subsection{Pre- and Post- state}
\label{sec:builtinconstructs}

\nodiff

Figure~\ref{fig:gram:oldandresult} summarizes the grammar extension of terms
with \lstinline|\old| and \lstinline|\result|.
\begin{figure}[htbp]
  \begin{cadre}
      \input{oldandresult_modern.bnf}
    \end{cadre}
    \caption{\protect\old and \protect\result in terms}
  \label{fig:gram:oldandresult}
\end{figure}

%%%%%%%%%%%%%%%%%%%%%%%%%%%%%%%%%%%%%%%%%%%%%%%%%%%%%%%%%%%%%%%%%%%%%%%%%%%%%%%

\subsection{Simple function contracts}
\label{sec:simplecontracts}

\nodiff

\difficultwhy{\notimplemented{\lstinline|assigns|}}{the implementation of a
  memory model}

%%%%%%%%%%%%%%%%%%%%%%%%%%%%%%%%%%%%%%%%%%%%%%%%%%%%%%%%%%%%%%%%%%%%%%%%%%%%%%%

\subsection{Contracts with named behaviors}
\label{subsec:behaviors}
\index{function behavior}\index{behavior}

\nodiff

%%%%%%%%%%%%%%%%%%%%%%%%%%%%%%%%%%%%%%%%%%%%%%%%%%%%%%%%%%%%%%%%%%%%%%%%%%%%%%%

\subsection{Memory locations and sets of terms}
\label{sec:locations}

\except{ranges and \notimplemented{set comprehensions} are limited in order to
  be finite}

Figure~\ref{fig:gram:locations} describes the grammar of sets of terms. There
are two differences with \acsl:
\begin{itemize}
\item ranges necessarily have lower and upper bounds;
\item a guard for each binder is required when defining set comprehension. The
  requested constraints for guards are the very same than the ones for
  quantifications.
\end{itemize}
  
\begin{figure}[htbp]
  \fbox{\begin{minipage}{0.97\textwidth}
      \input{loc_modern.bnf}
    \end{minipage}}
  \caption{Grammar for sets of terms}
\label{fig:gram:locations}
\end{figure}

\begin{notimplementedenv}
\begin{example}\label{ex:tset}
The set \lstinline*{ x | integer x; 0 <= x <= 10 && x % 2 == 0}* denotes the
set of even integers between 0 and 10.
\end{example}
\end{notimplementedenv}

\ifthenelse{\boolean{PrintImplementationRq}}{%
  \begin{notimplementedenv}
    Ranges are currently only supported in memory built-ins described in
    Section~\ref{subsec:memory},~\ref{sec:initialized} and~\ref{sec:dangling}.
  \end{notimplementedenv}

  \begin{example}
    The predicate \lstinline|\\valid(&t[0 .. 9])| is supported and denotes
    that the ten first cells of the array \lstinline|t| are valid. Writing the
    term \lstinline|&t[0 .. 9]| alone, outside any memory built-in, is not yet
    supported.
  \end{example}
}

%%%%%%%%%%%%%%%%%%%%%%%%%%%%%%%%%%%%%%%%%%%%%%%%%%%%%%%%%%%%%%%%%%%%%%%%%%%%%%%

\subsection{Default contracts, multiple contracts}
\nodiff

%%%%%%%%%%%%%%%%%%%%%%%%%%%%%%%%%%%%%%%%%%%%%%%%%%%%%%%%%%%%%%%%%%%%%%%%%%%%%%%
%%%%%%%%%%%%%%%%%%%%%%%%%%%%%%%%%%%%%%%%%%%%%%%%%%%%%%%%%%%%%%%%%%%%%%%%%%%%%%%
%%%%%%%%%%%%%%%%%%%%%%%%%%%%%%%%%%%%%%%%%%%%%%%%%%%%%%%%%%%%%%%%%%%%%%%%%%%%%%%

\section{Statement annotations}
\index{annotation}

%%%%%%%%%%%%%%%%%%%%%%%%%%%%%%%%%%%%%%%%%%%%%%%%%%%%%%%%%%%%%%%%%%%%%%%%%%%%%%%

\subsection{Assertions}
\label{sec:assertions}
\indextt{assert}
\nodiff

Figure~\ref{fig:gram:assertions} summarizes the grammar for assertions.
\begin{figure}[htbp]
  \begin{cadre}
    \input{assertions_modern.bnf}
  \end{cadre}
  \caption{Grammar for assertions}
  \label{fig:gram:assertions}
\end{figure}

%%%%%%%%%%%%%%%%%%%%%%%%%%%%%%%%%%%%%%%%%%%%%%%%%%%%%%%%%%%%%%%%%%%%%%%%%%%%%%%

\subsection{Loop annotations}
\label{sec:loop_annot}

\except{loop invariants lose their inductive nature}

Figure~\ref{fig:gram:loops} shows the grammar for loop annotations. There is no
syntactic difference with \acsl.
\begin{figure}[htbp]
  \begin{cadre}
    \input{loops_modern.bnf}
  \end{cadre}
  \caption{Grammar for loop annotations}
  \label{fig:gram:loops}
\end{figure}

\difficultswhy{\notimplemented{\lstinline|loop allocation|} and
  \notimplemented{\lstinline|loop assigns|}}{the implementation of a memory
  model}

\subsubsection{Loop invariants}

\index{invariant} The semantics of loop invariants is the same than the one
defined in \acsl, except that they are not inductive. More precisely, if one
does not take care of side effects (the semantics of specifications about side
effects in loop is the same in \eacsl than the one in \acsl), a loop invariant
$I$ is valid in \acsl if and only if:
\begin{itemize}
\item $I$ holds before entering the loop; and
\item if $I$ is assumed true in some state where the loop condition $c$ is also
  true, and if the execution of the loop body in that state ends normally at the
  end of the body or with a "continue" statement, $I$ is true in the resulting
  state.
\end{itemize}

In \eacsl, the same loop invariant $I$ is valid if and only if:
\begin{itemize}
\item $I$ holds before entering the loop; and
\item if the execution of the loop body in that state ends normally at the end
  of the body or with a "continue" statement, $I$ is true in the resulting
  state.
\end{itemize}

Thus the only difference with \acsl is that \eacsl does not assume that the
invariant previously holds when one checks that it holds at the end of the loop
body. In other words a loop invariant \lstinline|I| is equivalent to putting an
assertion \lstinline|I| just before entering the loop and at the very end of the
loop body.

\begin{example}
In the following, \lstinline|bsearch(t,n,v)| searches for element \lstinline|v|
in array \lstinline|t| between indices \lstinline|0| and \lstinline|n-1|.

\cinput{bsearch.c}

In \eacsl, this annotated function is equivalent to the following one since
loop invariants are not inductive.

\cinput{bsearch2.c}

\end{example}

\subsubsection{General inductive invariant}
\label{sec:generalized-invariants}

The syntax of this kind of invariant is shown
Figure~\ref{fig:advancedinvariants}.

\begin{figure}[htbp]
  \begin{cadre}
    \input{generalinvariants_modern.bnf}
  \end{cadre}
  \caption{Grammar for general inductive invariants}
  \label{fig:advancedinvariants}
\end{figure}

In \eacsl, a general inductive invariant may be written everywhere in a loop
body, and is exactly equivalent to writing an assertion.

%%%%%%%%%%%%%%%%%%%%%%%%%%%%%%%%%%%%%%%%%%%%%%%%%%%%%%%%%%%%%%%%%%%%%%%%%%%%%%%

\subsection{Built-in construct \texorpdfstring{\at}{\textbackslash{}at}}
\label{sec:at}\indexttbs{at}

\except{no forward references}

The construct \verb|\at(t,id)| (where \verb|id| is a regular \C label, a label
added within a ghost statement or a default logic label)
follows the same rule than its \acsl counterpart, except that a more restrictive
scoping rule must be respected in addition to the standard \acsl scoping rule:
\begin{itemize}
\item when evaluating \verb|\at(t,id)| at a propram point $p$, the program point
  $p'$ denoted by \verb|id| must be reached before $p$ in the program execution
  flow; and
\item when evaluating \verb|\at(t,id)|, for each \C left-value $x$ that
  contributes to the definition of a (non-ghost) logic variable involved in $t$,
  the equality \verb|\at(x,id) == \at(x,Here)| must hold, i.e. the value of $x$
  must not be modified between the program points \verb|id| and \verb|Here|.
\end{itemize}

Below, the first example illustrates the first constraint, whereas the second
example illustrates the second constraint.

\begin{example}
  In the following example, both assertions are accepted and valid in \acsl, but
  only the first one is accepted and valid in \eacsl since evaluating the term
  \verb|\at(*(p+\at(*q,Here)),L1)| at \verb|L2| requires to evaluate the term
  \verb|\at(*q,Here)| at \verb|L1|: that is forbidden since \verb|L1| is
  executed before \verb|L2|.  \cinput{at.c}
\end{example}

\begin{example}
  In the following example, the first assertion is supported, while the second
  one is not supported. Indeed, in the second assertion, the guard defining the
  logic variable \verb|u| depends on \verb|n| whose value is modified between
  \verb|L1| and \verb|L2|.
  \cinput{at_on-purely-logic-variables.c}
\end{example}

\ifthenelse{\boolean{PrintImplementationRq}}{%
  \begin{notimplementedenv}
    Any \lstinline|\\at| construct involving a logic variable whose definition
    depends on a \C variable is currently unsupported by plug-in \eacsl.

    \begin{example}
      The \lstinline|\\old| construct (special case of \lstinline|\\at|) of the
      following example is \emph{not yet} supported since the guard of the
      quantified variable \lstinline|i| depends on the \C variable \lstinline|n|
      in the definition of its upper bound.
      \cinput{at_on-purely-logic-variables_not-yet.c}
    \end{example}
  \end{notimplementedenv}
}

%%%%%%%%%%%%%%%%%%%%%%%%%%%%%%%%%%%%%%%%%%%%%%%%%%%%%%%%%%%%%%%%%%%%%%%%%%%%%%%

\subsection{Statement contracts}
\label{sec:statement_contract}
\index{statement contract}\index{contract}

\nodiff

Figure~\ref{fig:gram:stcontracts} shows the grammar of statement contracts.

\begin{figure}[htbp]
  \begin{cadre}
    \input{st_contracts_modern.bnf}
  \end{cadre}
  \caption{Grammar for statement contracts}
  \label{fig:gram:stcontracts}
\end{figure}

%%%%%%%%%%%%%%%%%%%%%%%%%%%%%%%%%%%%%%%%%%%%%%%%%%%%%%%%%%%%%%%%%%%%%%%%%%%%%%%
%%%%%%%%%%%%%%%%%%%%%%%%%%%%%%%%%%%%%%%%%%%%%%%%%%%%%%%%%%%%%%%%%%%%%%%%%%%%%%%
%%%%%%%%%%%%%%%%%%%%%%%%%%%%%%%%%%%%%%%%%%%%%%%%%%%%%%%%%%%%%%%%%%%%%%%%%%%%%%%

\section{Termination}
\label{sec:termination}
\index{termination}

\except{no \lstinline|terminates| clauses}

%%%%%%%%%%%%%%%%%%%%%%%%%%%%%%%%%%%%%%%%%%%%%%%%%%%%%%%%%%%%%%%%%%%%%%%%%%%%%%%

\subsection{Measure}
\nodiff

\subsection{\notimplemented{Integer measures}}
\label{sec:integermeasures}
\indexttbs{decreases}\indexttbs{variant}
\nodiff

%%%%%%%%%%%%%%%%%%%%%%%%%%%%%%%%%%%%%%%%%%%%%%%%%%%%%%%%%%%%%%%%%%%%%%%%%%%%%%%

\subsection{\notimplemented{General measures}}
\label{sec:generalmeasures}
\nodiff

%%%%%%%%%%%%%%%%%%%%%%%%%%%%%%%%%%%%%%%%%%%%%%%%%%%%%%%%%%%%%%%%%%%%%%%%%%%%%%%

\subsection{\notimplemented{Recursive function calls}}
\nodiff

%%%%%%%%%%%%%%%%%%%%%%%%%%%%%%%%%%%%%%%%%%%%%%%%%%%%%%%%%%%%%%%%%%%%%%%%%%%%%%%

\subsection{Non-terminating functions}
\absentwhy{whether a function is guaranteed to terminate if some predicate $p$
  holds is not a monitorable property}

\subsection{Measures and non-terminating functions}
\nodiff

%%%%%%%%%%%%%%%%%%%%%%%%%%%%%%%%%%%%%%%%%%%%%%%%%%%%%%%%%%%%%%%%%%%%%%%%%%%%%%%
%%%%%%%%%%%%%%%%%%%%%%%%%%%%%%%%%%%%%%%%%%%%%%%%%%%%%%%%%%%%%%%%%%%%%%%%%%%%%%%
%%%%%%%%%%%%%%%%%%%%%%%%%%%%%%%%%%%%%%%%%%%%%%%%%%%%%%%%%%%%%%%%%%%%%%%%%%%%%%%

\section{Logic specifications}
\label{sec:logicspec}
\index{logic specification}\index{specification}

\nodiff

Figure~\ref{fig:gram:logic} presents the grammar of logic definitions.

\begin{figure}[htbp]
  \fbox{\begin{minipage}{0.97\linewidth}\vfill \input{logic_modern.bnf}
    \vfill\end{minipage}}
  \caption{Grammar for global logic definitions}
\label{fig:gram:logic}
\end{figure}

%%%%%%%%%%%%%%%%%%%%%%%%%%%%%%%%%%%%%%%%%%%%%%%%%%%%%%%%%%%%%%%%%%%%%%%%%%%%%%%

\subsection{Predicate and function definitions}
\nodiff

%%%%%%%%%%%%%%%%%%%%%%%%%%%%%%%%%%%%%%%%%%%%%%%%%%%%%%%%%%%%%%%%%%%%%%%%%%%%%%%

\subsection{\notimplemented{Lemmas}}
\nodiff

Lemmas are verified before running the function \lstinline|main| but after
initializing global variables.

%%%%%%%%%%%%%%%%%%%%%%%%%%%%%%%%%%%%%%%%%%%%%%%%%%%%%%%%%%%%%%%%%%%%%%%%%%%%%%%

\subsection{\notimplemented{Inductive predicates}}
\label{sec:inductivepredicates}
\index{inductive predicates}
\experimental

\nodiff

Figure~\ref{fig:gram:inductive} presents the grammar of inductive predicates.

\begin{figure}[htbp]
  \fbox{\begin{minipage}{0.97\linewidth}\vfill \input{inductive_modern.bnf}
    \vfill\end{minipage}}
  \caption{Grammar for inductive predicates}
\label{fig:gram:inductive}
\end{figure}

Inductive predicates in all their generality are not monitorable. Therefore,
future versions of this document will restrict them syntactically (e.g., to
definite Horn clauses (\url{http://en.wikipedia.org/wiki/Horn_clause}) and/or
through semantic criteria.

%%%%%%%%%%%%%%%%%%%%%%%%%%%%%%%%%%%%%%%%%%%%%%%%%%%%%%%%%%%%%%%%%%%%%%%%%%%%%%%

\subsection{\notimplemented{Axiomatic definitions}}
\label{sec:axiomatic}
\experimental

\nodiff

Figure~\ref{fig:gram:logicdecl} presents the grammar of axiomatic definitions.

\begin{figure}[htbp]
  \fbox{\begin{minipage}{0.97\linewidth}\vfill \input{logicdecl_modern.bnf}
    \vfill\end{minipage}}
  \caption{Grammar for axiomatic declarations}
\label{fig:gram:logicdecl}
\end{figure}

Axiomatic definitions in all their generality are not monitorable. Therefore,
future versions of this document will restrict them syntactically and/or
through semantic criteria.

%%%%%%%%%%%%%%%%%%%%%%%%%%%%%%%%%%%%%%%%%%%%%%%%%%%%%%%%%%%%%%%%%%%%%%%%%%%%%%%

\subsection{\notimplemented{Polymorphic logic types}}
\label{sec:polym-logic-types}
\index{type!polymorphic}
\index{polymorphism}
\nodiff

%%%%%%%%%%%%%%%%%%%%%%%%%%%%%%%%%%%%%%%%%%%%%%%%%%%%%%%%%%%%%%%%%%%%%%%%%%%%%%%

\subsection{Recursive logic definitions}
\index{recursion}
\nodiff

%%%%%%%%%%%%%%%%%%%%%%%%%%%%%%%%%%%%%%%%%%%%%%%%%%%%%%%%%%%%%%%%%%%%%%%%%%%%%%%

\subsection{Higher-order logic constructions}
\label{sec:higherorder}

\experimental

\nodiff

Figure~\ref{fig:gram:higherorder} introduces new term constructs for
higher-order logic.

\begin{figure}[htp]
  \begin{cadre}
      \input{higherorder_modern.bnf}
    \end{cadre}
  \caption{Grammar for higher-order constructs}
\label{fig:gram:higherorder}
\end{figure}

{\highlightnotimplemented Abstractions are only implemented for extended
  quantifiers, such as the term \lstinline|\sum(1, 10, \lambda integer k; k)|.}

%%%%%%%%%%%%%%%%%%%%%%%%%%%%%%%%%%%%%%%%%%%%%%%%%%%%%%%%%%%%%%%%%%%%%%%%%%%%%%%

\subsection{\notimplemented{Concrete logic types}}
\label{sec:concrete-logic-types}
\index{type!concrete}
\experimental

\nodiff

Figure~\ref{fig:gram:logictype} introduces new constructs for defining logic
types and the associated new terms.

\begin{figure}[htp]
  \begin{cadre}
      \index{type!record}\index{type!sum}
      \input{logictypedecl_modern.bnf}
    \end{cadre}
  \caption{Grammar for concrete logic types and pattern-matching}
\label{fig:gram:logictype}
\end{figure}

%%%%%%%%%%%%%%%%%%%%%%%%%%%%%%%%%%%%%%%%%%%%%%%%%%%%%%%%%%%%%%%%%%%%%%%%%%%%%%%

\subsection{\notimplemented{Hybrid functions and predicates}}
\label{sec:logicalstates}
\index{hybrid!function}
\index{hybrid!predicate}
\nodiff

\difficultswhy{\notimplemented{Hybrid functions and predicates}}{the
  implementation of a memory model (or at least to support \lstinline|\\at|)}

%%%%%%%%%%%%%%%%%%%%%%%%%%%%%%%%%%%%%%%%%%%%%%%%%%%%%%%%%%%%%%%%%%%%%%%%%%%%%%%

\subsection{\notimplemented{Memory footprint specification: \texorpdfstring{\lstinline|reads|}{reads} clause}}
\label{sec:reads}
\experimental

\nodiff

Figure~\ref{fig:gram:logicreads} introduces \lstinline|reads| clauses.

\begin{figure}[htp]
  \begin{cadre}
      \input{logicreads_modern.bnf}
    \end{cadre}
  \caption{Grammar for logic declarations with \lstinline|reads| clauses}
\label{fig:gram:logicreads}
\end{figure}

\difficultswhy{\notimplemented{read clauses}}{the implementation of a memory
  model}

%%%%%%%%%%%%%%%%%%%%%%%%%%%%%%%%%%%%%%%%%%%%%%%%%%%%%%%%%%%%%%%%%%%%%%%%%%%%%%%

\subsection{\notimplemented{Specification Modules}}
\label{sec:specmodules}
\nodiff

%%%%%%%%%%%%%%%%%%%%%%%%%%%%%%%%%%%%%%%%%%%%%%%%%%%%%%%%%%%%%%%%%%%%%%%%%%%%%%%
%%%%%%%%%%%%%%%%%%%%%%%%%%%%%%%%%%%%%%%%%%%%%%%%%%%%%%%%%%%%%%%%%%%%%%%%%%%%%%%
%%%%%%%%%%%%%%%%%%%%%%%%%%%%%%%%%%%%%%%%%%%%%%%%%%%%%%%%%%%%%%%%%%%%%%%%%%%%%%%

\section{Pointers and physical adressing}
\label{sec:pointers}

\nodiff

Figure~\ref{fig:gram:memory} shows the additional constructs for terms and
predicates which are related to memory location.
\begin{figure}[htbp]
  \fbox{\begin{minipage}{0.97\linewidth}
      \input{memory_modern.bnf}
    \end{minipage}}
  \caption{Grammar extension of terms and predicates about memory}
\label{fig:gram:memory}
\end{figure}

%%%%%%%%%%%%%%%%%%%%%%%%%%%%%%%%%%%%%%%%%%%%%%%%%%%%%%%%%%%%%%%%%%%%%%%%%%%%%%%

\subsection{Memory blocks and pointer dereferencing}
\label{subsec:memory}
\label{sec:memory} % for correctness of \changeinsection in Changes
\nodiff

\difficultswhy{All memory-related built-in functions and predicates}{the
  implementation of a memory model}

%%%%%%%%%%%%%%%%%%%%%%%%%%%%%%%%%%%%%%%%%%%%%%%%%%%%%%%%%%%%%%%%%%%%%%%%%%%%%%%

\subsection{Separation}\label{sec:separation}
\nodiff

\difficultwhy{\lstinline|\\separated|}{the implementation of a memory model}

%%%%%%%%%%%%%%%%%%%%%%%%%%%%%%%%%%%%%%%%%%%%%%%%%%%%%%%%%%%%%%%%%%%%%%%%%%%%%%%

\subsection{\notimplemented{Dynamic allocation and deallocation}}
\label{sec:alloc-dealloc}
\nodiff

\difficultswhy{All these constructs}{the implementation of a memory model}

Figure~\ref{fig:gram:allocation} introduces grammar for dynamic allocations and
deallocations.

\begin{figure}[htp]
  \begin{cadre}
      \input{allocation_modern.bnf}
    \end{cadre}
  \caption{Grammar for dynamic allocations and deallocations}
\label{fig:gram:allocation}
\end{figure}

%%%%%%%%%%%%%%%%%%%%%%%%%%%%%%%%%%%%%%%%%%%%%%%%%%%%%%%%%%%%%%%%%%%%%%%%%%%%%%%
%%%%%%%%%%%%%%%%%%%%%%%%%%%%%%%%%%%%%%%%%%%%%%%%%%%%%%%%%%%%%%%%%%%%%%%%%%%%%%%
%%%%%%%%%%%%%%%%%%%%%%%%%%%%%%%%%%%%%%%%%%%%%%%%%%%%%%%%%%%%%%%%%%%%%%%%%%%%%%%

\section{\notimplemented{Sets and lists}}
\index{location}

\subsection{\notimplemented{Finite sets}}
\nodiff

\subsection{\notimplemented{Finite lists}}
\nodiff

Figure~\ref{fig:gram:list} shows the notations for built-in lists.
\begin{figure}[t]
  \begin{cadre}
      \begin{syntax}
  term ::=  { "[|" "|]" } ; empty list
       | { "[|" term ("," term)* "|]" } ; list of elements
       | { term "^" term } ; list concatenation (overloading bitwise-xor 
                         ; operator)
       | { term "*^" term } ; list repetition
\end{syntax}

    \end{cadre}
  \caption{Notations for built-in list datatype}
\label{fig:gram:list}
\end{figure}

%%%%%%%%%%%%%%%%%%%%%%%%%%%%%%%%%%%%%%%%%%%%%%%%%%%%%%%%%%%%%%%%%%%%%%%%%%%%%%%
%%%%%%%%%%%%%%%%%%%%%%%%%%%%%%%%%%%%%%%%%%%%%%%%%%%%%%%%%%%%%%%%%%%%%%%%%%%%%%%
%%%%%%%%%%%%%%%%%%%%%%%%%%%%%%%%%%%%%%%%%%%%%%%%%%%%%%%%%%%%%%%%%%%%%%%%%%%%%%%

\section{\notimplemented{Abrupt termination}}
\label{sec:abrupt}
\index{abrupt termination}

\nodiff

Figure~\ref{fig:gram:abrupt} shows the grammar of abrupt terminations.

\begin{figure}[htbp]
  \begin{cadre}
    \input{abrupt_modern.bnf}
  \end{cadre}
  \caption{Grammar of contracts about abrupt terminations}
  \label{fig:gram:abrupt}
\end{figure}

%%%%%%%%%%%%%%%%%%%%%%%%%%%%%%%%%%%%%%%%%%%%%%%%%%%%%%%%%%%%%%%%%%%%%%%%%%%%%%%
%%%%%%%%%%%%%%%%%%%%%%%%%%%%%%%%%%%%%%%%%%%%%%%%%%%%%%%%%%%%%%%%%%%%%%%%%%%%%%%
%%%%%%%%%%%%%%%%%%%%%%%%%%%%%%%%%%%%%%%%%%%%%%%%%%%%%%%%%%%%%%%%%%%%%%%%%%%%%%%

\section{\notimplemented{Dependencies information}}
\label{sec:func-dep}
\experimental

\nodiff

Figure~\ref{fig:gram:dep} shows the grammar for dependencies information.

\begin{figure}[htp]
  \begin{cadre}
      \input{dependencies_modern.bnf}
    \end{cadre}
  \caption{Grammar for dependencies information}
\label{fig:gram:dep}
\end{figure}

%%%%%%%%%%%%%%%%%%%%%%%%%%%%%%%%%%%%%%%%%%%%%%%%%%%%%%%%%%%%%%%%%%%%%%%%%%%%%%%
%%%%%%%%%%%%%%%%%%%%%%%%%%%%%%%%%%%%%%%%%%%%%%%%%%%%%%%%%%%%%%%%%%%%%%%%%%%%%%%
%%%%%%%%%%%%%%%%%%%%%%%%%%%%%%%%%%%%%%%%%%%%%%%%%%%%%%%%%%%%%%%%%%%%%%%%%%%%%%%

\section{\notimplemented{Data invariants}}
\label{sec:invariants}
\index{data invariant}\index{global invariant}\index{type invariant}
\index{invariant!data}\index{invariant!global}\index{invariant!type}

\nodiff

Figure~\ref{fig:gram:datainvariants} summarizes grammar for declarations of data
invariants.
\begin{figure}[htbp]
  \fbox{\begin{minipage}{0.97\linewidth}
      \input{data_invariants_modern.bnf}
    \end{minipage}}
  \caption{Grammar for declarations of data invariants}
\label{fig:gram:datainvariants}
\end{figure}

\unefficient{strong invariants}

%%%%%%%%%%%%%%%%%%%%%%%%%%%%%%%%%%%%%%%%%%%%%%%%%%%%%%%%%%%%%%%%%%%%%%%%%%%%%%%

\subsection{Semantics}
\nodiff

%%%%%%%%%%%%%%%%%%%%%%%%%%%%%%%%%%%%%%%%%%%%%%%%%%%%%%%%%%%%%%%%%%%%%%%%%%%%%%%

\subsection{\notimplemented{Model variables and model fields}}
\label{sec:model}
\index{model}

\nodiff

Figure~\ref{fig:gram:model} summarizes the grammar for declarations of model
variables and fields.
\begin{figure}[htbp]
  \fbox{\begin{minipage}{0.97\linewidth}
      \input{model_modern.bnf}
    \end{minipage}}
  \caption{Grammar for declarations of model variables and fields}
\label{fig:gram:model}
\end{figure}

%%%%%%%%%%%%%%%%%%%%%%%%%%%%%%%%%%%%%%%%%%%%%%%%%%%%%%%%%%%%%%%%%%%%%%%%%%%%%%%
%%%%%%%%%%%%%%%%%%%%%%%%%%%%%%%%%%%%%%%%%%%%%%%%%%%%%%%%%%%%%%%%%%%%%%%%%%%%%%%
%%%%%%%%%%%%%%%%%%%%%%%%%%%%%%%%%%%%%%%%%%%%%%%%%%%%%%%%%%%%%%%%%%%%%%%%%%%%%%%

\section{Ghost variables and statements}
\label{sec:ghost}
\index{ghost}

\nodiff

Figure~\ref{fig:gram:ghost} summarizes the grammar for ghost statements which is
the same than the one of \acsl.
\begin{figure}[htbp]
  \fbox{\begin{minipage}{0.98\linewidth}
      \input{ghost_modern.bnf}
    \end{minipage}}
  \caption{Grammar for ghost statements}
\label{fig:gram:ghost}
\end{figure}

%%%%%%%%%%%%%%%%%%%%%%%%%%%%%%%%%%%%%%%%%%%%%%%%%%%%%%%%%%%%%%%%%%%%%%%%%%%%%%%

\subsection{\notimplemented{Volatile variables}}
\label{sec:volatile-variables}
\indextt{volatile}

Figure~\ref{fig:gram:volatile} summarizes the grammar for volatile constructs.
\begin{figure}[htp]
  \begin{cadre}
      \input{volatile-gram_modern.bnf}
    \end{cadre}
  \caption{Grammar for volatile constructs}
\label{fig:gram:volatile}
\end{figure}

%%%%%%%%%%%%%%%%%%%%%%%%%%%%%%%%%%%%%%%%%%%%%%%%%%%%%%%%%%%%%%%%%%%%%%%%%%%%%%%
%%%%%%%%%%%%%%%%%%%%%%%%%%%%%%%%%%%%%%%%%%%%%%%%%%%%%%%%%%%%%%%%%%%%%%%%%%%%%%%
%%%%%%%%%%%%%%%%%%%%%%%%%%%%%%%%%%%%%%%%%%%%%%%%%%%%%%%%%%%%%%%%%%%%%%%%%%%%%%%

\section{Initialization and undefined values}
\label{sec:initialized}
\nodiff

\difficultwhy{\lstinline|\\initialized|}{the implementation of a memory model}

\ifthenelse{\boolean{PrintImplementationRq}}{%
  \begin{notimplementedenv}
    The \framac plug-in \eacsl does not support labels as arguments of
    \lstinline|\\initialized|.
  \end{notimplementedenv}
}

\section{\notimplemented{Dangling pointers}}
\label{sec:dangling}
\nodiff

\difficultwhy{\notimplemented{\lstinline|\\dangling|}}{the implementation of a
  memory model}

%%%%%%%%%%%%%%%%%%%%%%%%%%%%%%%%%%%%%%%%%%%%%%%%%%%%%%%%%%%%%%%%%%%%%%%%%%%%%%%
%%%%%%%%%%%%%%%%%%%%%%%%%%%%%%%%%%%%%%%%%%%%%%%%%%%%%%%%%%%%%%%%%%%%%%%%%%%%%%%
%%%%%%%%%%%%%%%%%%%%%%%%%%%%%%%%%%%%%%%%%%%%%%%%%%%%%%%%%%%%%%%%%%%%%%%%%%%%%%%

\section{Well-typed pointers}
\label{sec:typedpointers}
\absentwhy{it would require the implementation of a \C type system at runtime}

%%%%%%%%%%%%%%%%%%%%%%%%%%%%%%%%%%%%%%%%%%%%%%%%%%%%%%%%%%%%%%%%%%%%%%%%%%%%%%%
%%%%%%%%%%%%%%%%%%%%%%%%%%%%%%%%%%%%%%%%%%%%%%%%%%%%%%%%%%%%%%%%%%%%%%%%%%%%%%%
%%%%%%%%%%%%%%%%%%%%%%%%%%%%%%%%%%%%%%%%%%%%%%%%%%%%%%%%%%%%%%%%%%%%%%%%%%%%%%%

\section{Preprocessing for ACSL}
\nodiff
