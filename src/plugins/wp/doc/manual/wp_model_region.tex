
\section{The Region Memory Model}
\label{sec:region-model}

The Region memory model dispatches each pointer to a different memory model.
For instance, an \texttt{int*} pointer that is always accessed through
\texttt{int*} accesses (with possible aliases) need only to rely on the Typed
memory model. On the contrary, a value containing an union whose fields are
accessed with heterogeneous types shall be handled using the Bytes memory model.

\paragraph{Usage.} To activate this model, simply use '\texttt{-wp-model Region}'.
It will automatically launch the plugin Region, even without the usage of
'\texttt{-region}'.

\paragraph{Addresses.} Each address in the Region memory model is a pair of
the representation of its physical address (computed by the Bytes memory
model) and the region in which it is included (over-approximation).

\paragraph{Plugin Region.}
The plugin Region statically computes for each function a static region
map that associates a node (or region) to expressions of pointers. Each
node of the map represents a set of pointer's addresses and includes
the computation of an over-approximation of the list of access types
(e.g. write 32bits int, read 64bits float, cast 16bits int, etc\dots).

\paragraph{Bytes memory model \& Region plugin.}
The Region memory model uses the region map produced by the Region
plugin to statically associate each pointer to a region. Then, for each
pointer whose region's accesses are not heterogeneous, the Region memory
model behaves the same way as the Typed memory model ; whereas for all
regions with heterogeneous accesses, the Region memory model sends off all
its pointers to the Bytes memory model.

\paragraph{Limitations.} The Region memory model is provided with the
following current limitations:
\begin{itemize}
    \item Function calls are not yet implemented in the Region plugin.
    \item A structure copy might be rejected by the model if the structure
    contains fields with pointers.
    \item Logic is not yet included and only C code is currently analyzed by
    the Region plugin. Accesses (read, write, and cast) happening only inside
    ACSL annotations are not included in the analysis. As a consequence,
    developers and formalists shall add (possibly ghost) C code for field
    accesses that only appear inside ACSL annotations.
\end{itemize}
